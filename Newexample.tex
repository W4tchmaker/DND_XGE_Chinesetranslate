\documentclass[letterpaper,10pt,twoside,twocolumn,openany]{dndbook}

\usepackage[utf8]{inputenc}
\usepackage{xeCJK}
\usepackage{lipsum}
\usepackage{listings}
\usepackage{indentfirst}
\parindent=20pt

\lstset{%
  basicstyle=\ttfamily,
  language=[LaTeX]{TeX},
}
\begin{document}
\chapter{简介Introduction}
\par{在熙熙攘攘的深水城下,身为罪犯之王的眼魔监视着  所有的人事物——至少它自己是这么认为的。人称珊娜萨Xanathar的这一古怪存在觉得自己可以收集龙与地下城多元宇宙中万事万物的情报。眼魔它想要知道一切!不过无论眼魔再如何研习与蒐集宝物,它在多元宇宙中最珍贵的财产仍然是它的金鱼,斯尔嘉Sylgar。}
\par{《珊娜萨的万事指南Xanathar’s Guide to Everything》作为第五版D\&D的首个大型规则扩展为该游戏提供了大量的新选项。珊娜萨或许无法实现其了解万物的梦想,但本书确实深入研究了游戏的每个主要部分:冒险者,以及他们经历的冒险本身和他们行使的魔法。}
\section{使用本书 Using This Book }
\par{本书为玩家和地下城主而著。无论你是在被遗忘国度里冒险,还是在另一个官方D\&D设定中,又或者在你自己创作的世界中,本书所提供的选项都可以为战役世界作加成增进。这些选项建立在《玩家手册》《怪物图鉴》和《城主指南》中所包含的官方规则基础之上。你可以将本书看作是原三宝书的附带。它以三宝书的内容为基础,探索这些书刊里最初所述内容的走向。这里的内容不是D\&D战役中必需的东西(这不算第四本核心规则书),但我们希望它能为你提供一些享受该游戏的新方式。}
\par{第1章提供了《玩家手册》中所述角色选线的扩展。第2章是个DM的工具包,它提供了一些运行游戏和设计冒险的新资源,而所有这些需要以《怪物图鉴》和《城主指南》为基础进行构建。第3章则为玩家角色和施法怪物们提供了一些新的法术。}
\par{附录A提供了关于运作与D\&D冒险者联盟Adventurers League类似的共享战役之指引,而附录B包含的许多表格则可以帮你快速地为自己的D\&D故事角色生成名字。}
\par{你阅读本书中诸多选项时,还会遇到来自珊娜萨的观察。你的阅读体验或许会随着眼魔那翻滚的思维将你带去某些熟悉又新鲜的游戏中。祝你旅途愉快!}
\begin{paperbox}{破解奥秘Unearthed Arcana}
本书的大部分内容都曾在破解奥秘中出现。破解奥秘是我们发布的一系列在线文章,用于探索那些或许能应用于游戏的规则。那些未能引起引起粉丝共鸣的破解奥秘内容会被暂时搁置。在下列篇章中的破解奥秘材料都是广受欢迎的部分,而这些材料也已经在千万人的反馈中作了进一步的提炼才得以正式发布。
\end{paperbox}
\clearpage
\section{核心规则The Core Rules }
\par{本书基于三本核心规则书的规则。游戏会频繁用到《玩家手册》第7~10章的规则,包括:“属性值应用”“冒险”“战斗”“施法”,以及其附录A中包括的隐形和倒地等重要状态。你不需要死记这些规则,不过最好能再需要它们时知道要到哪里去查询。}
\par{如果你是DM,你还要知道如何在《城主指南》中查找相应的内容,特别是一些魔法物品的运作方式(件该书第7章)。关于如何阅读怪物资料板的指引可以参见《怪物图鉴》的简介。}
\subsection{城主裁决规则  The DM Adjudicates the}
\par{一条凌驾一切的规则:DM是所有游戏中运作规则的最终裁决者。}
\par{规则是D\&D能成为游戏的重要部分,其作用不仅仅是提升故事体验。游戏规则旨在帮助组织,甚至启发D\&D战役的行进。规则是工具,而我们希望这些工具能够物尽其用。但不管工具有多好,它们都需要由一名DM指引再由玩家们躬行才能发挥其作用。}
\par{DM是其中关键。D\&D战役中可能会出现许多意想不到的事件,既定的规则无法尽善尽美的解释每一种意外。试图面面俱到的规则往往会使游戏变得步履维艰。而严格限制角色们所作所为的想法与D\&D的开放性理念相违背。该游戏应包含的宗旨为:它应为DM构建规则奠定基础,并协助DM作为桥梁来判定规则与万物之间关联或非关联的性质。}
\subsection{谨记的十条规则 Ten Rules to Remember }
\par{核心规则书中有一些规则有时会难倒新手玩家或DM,以下十条便是类似这样的规则范例。留意这几条规则可以帮你理解本书中的选项。}
\subsubsection{特殊优先于通用 Exceptions Supersede General Rules }
\par{游戏各部分都对应了一些通用的规则。例如,战斗规则规定近战武器攻击使用力量值,远程武器攻击使用敏捷值。这就是一项通用规则,而该通用规则在游戏中未有额外说明的前提下都属有效。}
\subsubsection{向下取整 Round Down}
\par{你在游戏中对数字进行乘除运算时,如果其结果带分数,则即使其分数部分等于或大于一半,最终结果也要向下取整。}
\subsubsection{优势与劣势 Advantage and Disadvantage}
\par{在一次掷骰中就算有多项因素让你获得优势或劣势,你都只获得一次优势或劣势,而当你同时获得优势和劣势时,则本次掷骰你既没有优势也没有劣势。}
\subsubsection{不同效应混合 Combining Different Effects}
\par{不同的游戏效应可能会同时生效于同一目标。例如,可能会有两项不同的增益为你的护甲等级提供加值。不过,若两项或更多具有相同名字的效应同时作用,则在效应时间重叠时将只有其中一项得以生效(若两项效应不一样时则其中较强一项得以生效)。例如,当你在仍带有祝福术bless效应时再被施展一发祝福术bless,则你身上仍然只生效一个该法术的增益。与之相似,如果你身处超过一个“防护灵光”的范围内,则你也只能获得其中最强力的一项增益。}
\subsubsection{反应时机 Reaction Timing  }
\par{一些特定的游戏特性会让你执行一项称为反应的特殊动作以应对某些事件。发动借机攻击和施展护盾术shield是两种最典型的反应。如果你不确定一项反应与其触发条件的联系,则请遵从以下规则:除非反应的描述另有说明,否则该反应发生在触发事件完成后。}
\par{你执行一项反应后,直至下个你自己回合开始前都无法再次执行反应。}
\subsubsection{抗性与易伤 Resistance and Vulnerability }
\par{伤害生效时其调整值作用顺序如下:(1)伤害相关的所有免疫,(2)伤害相关的所有加值或罚值,(3)一项伤害相关的抗性,以及(4)一项伤害相关的易伤。}
\par{就算有多于一项作用让你获得同一类型伤害的抗性,也只有一份伤害抗性得以生效。计算易伤时也是如此。}
\subsubsection{熟练加值 Proficiency Bonus }
\par{你的熟练加值可以生效于某掷骰时,即使有多项因素让你得以应用熟练加值,你也只能在该掷骰中使用一次该加值。此外,即使有多项因素让你得以让加值翻倍或减半,你也只能在加值生效前进行一次相应运算。不管是经过乘除运算还是保留其原值的加值,均只生效于对应的一次掷骰中。}
\subsubsection{附赠动作法术 Bonus Action Spells  }
\par{施展施法时间为1附赠动作的法术时,你在同一回合中之前和之后都必须未施展任何其他法术,不过此间你仍可以施展施法时间为1动作的戏法。}
\subsubsection{专注 Concentration }
\par{当你开始施展的法术,或开始使用的特殊能力需要专注时,你正对其他效应进行的专注将立即终止。}
\subsubsection{临时生命值 Temporary Hit Points }
\par{临时生命值无法累加。若你在拥有临时生命值时再次获得临时生命值效应,除非相应的游戏特性另有说明,否则你不能将其数值相加。不过你可以自行选择保留哪项临时生命值。}
\clearpage
\clearpage
\chapter{野蛮人 BARBARIAN}
\begin{quote}
我见证了野蛮人在战斗中不屈不挠的表现,令我我好奇的是——他们狂暴的核心力量源于什么?——大法师赛雷特(Seret)
\end{quote}
\par{普通人所展现出的愤怒与野蛮人的狂暴在本质上相同,就像和煦微风与狂风骤雨的关系。野蛮人的驱动力来自超越纯粹的情感,使它表现得更加可怕。愤怒的动力可能完全源于心灵,又或者来自与动物精魂的链接,处于狂暴中的野蛮人建立联系能够展现出超自然的力量与耐力。爆发是暂时的,但当它持续时,就会控制野蛮人的身体和心灵,驱使野蛮人忽略危险和伤害,直到最后一名敌人倒下。}
\par{扮演野蛮人角色是很诱人的事,能够简单直白地塑造一个经典形象——残暴、通常还很蠢、在其他人小心翼翼时直接莽上去。但不是这世界上所有的野蛮人都如出一辙,所以你当然可以自己掂量着扮演。无论哪种方式,考虑添加一些修饰,使你的野蛮人独具特色;参考以下章节以获取灵感。}
\section{个人图腾 PERSONAL TOTEMS}
\par{野蛮人倾向于轻装前行,很少携带个人物品或其他不必要的装备。他们中的少数人随身携带的物品通常包括具有特殊意义的小东西。一枚个人图腾正是这样意义重大的东西——因为它有一个神秘起源,或者与角色生命中的某个重要瞬间相关,也许是有关野蛮人过去的一段回忆,或预示前方等待着他的命运。}
\par{这种个人图腾可能与野蛮人的动物精魂有关,或者实际上可能是该动物图腾,但这种联系并非必然。例如,一个有熊图腾精魂的野蛮人,仍然可以把鹰的羽毛作为个人图腾。}
\setthemecolor[gray]
\begin{dndtable}[cX]
\textbf{d6} & \textbf{图腾} \\
1 & 一条独狼身上的一簇毛,你在一次捕猎中和它成了朋友。 \\
2 & 智慧的萨满给你了三只鹰的羽毛,告诉你它们将决定你的命运。 \\
3 & 用年轻穴熊的爪子做成的项链,你还是个孩子的时候单杀了它。 \\
4 & 装着三块石头的小皮袋,代表了你的祖先。 \\
5 & 你杀死的第一只野兽的几块骨头,用彩色毛绳绑在一起。 \\
6 & 你的动物精魂形状的一块鸡蛋大小的石头,有一天出现在你的腰袋里。 \\
\end{dndtable}
\section{}
\end{document}
