%!TEX root = Mainbook.tex
\clearpage
\chapter{野蛮人 Barbarian}
\begin{quote}
\emph{
我见证了野蛮人在战斗中不屈不挠的表现,令我好奇的是,他们狂暴的核心力量源于什么?——大法师Seret}
\end{quote}

普通人所展现出的愤怒与野蛮人的狂暴本质上相同,两者的关系如同和煦微风与呼啸疾风。只是野蛮人的驱动力来自超越纯粹的情感,使它表现更加可怕。无论这愤怒是源于心灵,还是来自于动物精魂的链接,狂暴的野蛮人都能表现出非同凡响的力量和耐力。爆发尽管短暂,但它能持续控制野蛮人的身心,就算身负再多伤痛,都能奋战至最后一名敌人倒下倒下。

扮演野蛮人角色是件很诱人的事,能够简单直白地塑造经典形象——残暴、通常还很蠢、在其他人小心翼翼时直接莽上去。但不是这世界上所有的野蛮人都如出一辙,所以你当然可以自己掂量着扮演。无论哪种方式,考虑添加一些修饰,使你的野蛮人独具特色;参考以下章节以获取灵感。
\section{个人图腾 Personal Totems}野蛮人倾向于轻装前行,很少携带个人物品或其他不必要的装备。他们中的少数人随身携带的物品通常包括具有特殊意义的小东西。一枚个人图腾正符合意义重大的描述——因为它的神秘起源,并与角色生命中的某个重要瞬间相关,也许是有关野蛮人过去的一段回忆,或预示前方等待着他的命运。

这种个人图腾可能与野蛮人的动物精魂有关,或者实际上可能就是他的动物图腾,但这种联系并非必然。例如,一个有熊图腾精魂的野蛮人,仍然可以把鹰的羽毛作为个人图腾。
\setthemecolor[Phbclasstable]
\begin{dndtable}[cX]
\textbf{d6} & \textbf{图腾} \\
1 & \emph{一条独狼身上的一簇毛,你在一次捕猎中和它成了朋友。 }\\ 
2 & \emph{智慧的萨满给你了三只鹰的羽毛,告诉你它们将决定你的命运。 }\\ 
3 & \emph{用年轻穴熊的爪子做成的项链,你还是个孩子的时候单杀了它。 }\\ 
4 & \emph{装着三块石头的小皮袋,代表了你的祖先。 }\\ 
5 & \emph{你杀死的第一只野兽的几块骨头,用彩色毛绳绑在一起。 }\\ 
6 & \emph{你的动物精魂形状的一块鸡蛋大小的石头,有一天出现在你的腰袋里。 }\\ 
\end{dndtable}
\section{纹身 Tattoos}许多野蛮人都用纹身装饰自己,而每个纹身都代表着纹身者或者其先祖生命中的重要时刻,又或者象征着感情的宣泄和人生的态度。和个人图腾一样,野蛮人的纹身可能与动物精魂有关,又或者毫无关联。
野蛮人的每个纹身都诉说着这个人的身份,如果你的角色身着纹身,那这些纹身外观如何,又有何含义?
\begin{dndtable}[cX]
\textbf{d6} & \textbf{纹身} \\
1 & \emph{你在背上展开着雄鹰的羽翼。 }\\ 
2 & \emph{你手背上铭刻着穴居熊的利爪。 }\\ 
3 & \emph{你氏族的符号沿着你的手臂以微小的展示。 }\\ 
4 & \emph{你的腰后是墨染的麋鹿角。}\\ 
5 & \emph{你的武器和手上画着动物精魂的图案。 }\\ 
6 & \emph{你背上的狼眼,帮你洞察和抵御邪恶的灵魂。 }\\ 
\end{dndtable}
\section{迷信 Superstitions}野蛮人对生活的理解各不相同,有些信仰神明,有些从自然后期和他们遇到的动物上追寻神祇。这些野蛮人相信世上的动物和植物内里栖息着灵魂,并会向这些灵魂寻求预兆和力量。

其他野蛮人只相信手中的铁和心中的血。看不见的世界对他们而言毫无意义。取而代之,他们效仿猎杀的野兽,依靠自己的感官以在世间求生。
\begin{dndtable}[cX]
\textbf{d6} & \textbf{迷信} \\ 
1 & \emph{惊扰死者会让你麻烦缠身。 }\\ 
2 & \emph{别信法师,他们都些是恶魔,特别是像好人的。 }\\ 
3 & \emph{矮人没种,和死人没啥区别,老窝在地下。 }\\ 
4 & \emph{魔法总是会惹祸,睡觉时也要与最近的魔法物品离上十尺。 }\\ 
5 & \emph{走进坟场一定要带着银制品,免得鬼魂附体。 }\\ 
6 & \emph{要有精灵直视你,他是在试着读你的脑袋。}\\ 
\end{dndtable}

\section{原始道途 Primal Paths}第3级时,野蛮人将获得其原始道途特性,下列选项给出了在玩家手册提供的道途以外的额外道途选项:先祖守卫、风暴先驱和狂热者。
\subsection{先祖守卫 Path of the Ancestral Guardian}一些野蛮人尊重他们先祖的文化。这些人认为,过去战士们的果敢灵魂,可以引导并保护生者。当一个野蛮人追随这条道路时,他会联系精神世界,以寻求先祖的帮助。

选择先祖守卫的野蛮人的战力将大为提升,以保护他们的部族和盟友。为了更加纪念自己的先祖,选择这条道路的野蛮人将会在自己身上精心绘制纹身。而这些纹身往往代表着与凶恶的野兽和其他凶猛敌人之间曾展开的波澜壮阔的史诗。
\begin{dndtable}[cX]
\textbf{等级} & \textbf{特性} \\ 
3rd & \emph{先祖护卫 Ancestral Protectors }\\ 
6th & \emph{精魂之盾 Spirit Shield }\\ 
10th & \emph{问道精魂 Consult the Spirits }\\ 
14th & \emph{先祖复仇 Vengeful Ancestors}\\ 
\end{dndtable}
\subsubsection{先祖护卫 Ancestral Protectors}第3级选择该道途起,精魂战士会在你狂暴时出现。若你处于狂暴,你的回合内你攻击命中的第一个生物会成为精魂战士的目标,其攻击也受到阻碍。直到你的下一个回合开始,目标对你以外的目标的攻击检定具有劣势,若目标命中了除你以外的生物,该生物对那次攻击有伤害抗力。狂暴结束时,附加在敌人身上的效应中止。

\subsubsection{精魂之盾 Spirit Shield}第6级起,帮助你的卫士之魂会为你统御的盟友提供超自然保护。若你处于狂暴,而一个处于你30尺内对你可见的其他生物受到伤害,你可以用一个反应来将伤害降低2d6。

当你达到这该职业的更高等级,你可以防止更多的伤害,10级时增为3d6,14级时增为4d6。
\subsubsection{问道精魂 Consult the Spirits}第10级时,你得到和你的先祖之魂进行交流的能力。若你这样做,你可以不消耗法术位和材料成分使用一次\emph{卜筮术 augury}或\emph{鹰眼术 clairvoyance}。这个法术会在指定位置创造一个无形的先祖之魂取代法术描述里的半透明球体。感知是你的施法主属性。

你必须完成一次长休或短休才能再次使用此特性。
\subsubsection{先祖复仇 Vengeful Ancestors}第14级起,你的先祖之魂将强大到足以对伤害你所保护之人的敌人做出打击。当你使用精魂之盾降低一次攻击的伤害时,攻击者受到与精魂之盾所降低伤害相等量的力场伤害。

\subsection{风暴先驱 Path of the Storm Herald}所有野蛮人都内含愤怒。他们的狂暴让他们得到了强大的力量,耐力,和速度。而那些追随风暴先驱道途的野蛮人则学会了将他们的狂暴转化为环绕着自身的原初魔法灵光。当处于狂暴之下,这条道途的野蛮人深入自然,创造出强大的魔法效果。

典型的精英风暴先驱,会和德鲁伊、游侠以及其他发誓保护自然的人一起受训。但也有一些会在被暴风摧毁的房屋中,在极寒的世界冰原上,又或者灼热的沙漠最深处磨砺自己。
\begin{dndtable}[cX]
\textbf{等级} & \textbf{特性} \\ 
3rd & \emph{愤怒风暴 Storm of Fury }\\ 
6th & \emph{风暴之魂 Storm Soul }\\ 
10th & \emph{风暴之盾 Shield of the Storm }\\ 
14th & \emph{狂怒风暴 Raging Storm}\\ 
\end{dndtable}
\subsubsection{愤怒风暴 Storm of Fury}第3级选择此道途起,若你处于狂暴,你散发出一个10尺半径的灵光,但无法穿透全掩护。

你的灵光拥有一个会在你进入狂暴时所激发的效果,你在每个自己的回合也可以用一个附赠动作来激发该效果。从下述选项中选择一个:沙漠,海洋,冻原,此灵光的效果基于你选择的环境类型。每当你该职业的级别提升时,你可以改变所选择环境。

如果你的光环效果需要进行豁免,DC值为8+你的熟练加值+你的体质调整值。
\subparagraph{沙漠Desert}激发该效果时,你的灵光中所有其他生物都会受到2点火焰伤害。当你达到该职业的更高等级,伤害增加。在5级时增为3,10级时增为4,15级时增为5,20级时增为6。
\subparagraph{海洋Sea}激发该效果时,你可以选择一个处于灵光中对你可见的其他生物。目标必须进行一次敏捷豁免,豁免失败者将受到1d6闪电伤害,豁免成功者受到的伤害减半。当你达到该职业的更高等级,伤害增加——10级时增为2d6,15级时增为3d6,20级时增为4d6。
\subparagraph{冻原Rundra}激发该效果时,灵光中每个你由选择的生物获得2点临时生命,这是因为寒冰精魂使它惯于受苦。当你达到该职业的更高等级,临时生命增加——5级时增为3,10级时增为4,15级时增为5,20级时增为6。

\subsubsection{风暴之魂Storm Soul}第6级时,风暴在你不激发灵光时也能给你带来增益。增益取决于你为风暴灵光选择的的环境类型。
\subparagraph{沙漠Desert}你得到对火焰伤害的抗性,并不会受到极端高温的效果影响(见城主手册)。此外,作为一个动作,你可以触摸一个没有被他人穿戴或携带的易燃物体,并且点燃它。
\subparagraph{海洋Sea}你得到对闪电伤害的抗性,并能在水下呼吸。你还获得了30尺的游泳速度。
\subparagraph{冻原Tundra}你得到对寒冷伤害的抗性,并不会受到极端低温的效果影响(见城主手册)。此外,作为一个动作,你可以触摸水并且把5立方尺的水冻结成冰,这种冰将会在1分钟后融化。如果有生物处于该立方空间中,你的行动将会失败。
\subsubsection{风暴之盾 Shield of the Storm}第10级时,你学会了运用你的风暴的掌控力来保护你的盟友。处于你灵光内的、由你选择的生物得到你的风暴之魂特性所带来的伤害抗力。
\subsubsection{狂怒风暴 Raging Storm}第14级时,你能引导的风暴之力变得更加强大,足以打击你的敌人。效果取决于你为风暴灵光选择的的环境类型。
\subparagraph{沙漠 Desert}若处于你灵光中的生物攻击命中你,你可以立刻用你的反应迫使该生物进行一次敏捷豁免,豁免失败者将受到等于你野蛮人等级一半数值的火焰伤害。
\subparagraph{海洋 Sea}当处于你灵光中的生物被你攻击击中,你可以用你的反应迫使该生物进行一次力量豁免,豁免失败者将如同被波浪拍打般被击倒在地,变为俯卧状态。
\subparagraph{冻原 Tundra}当你的风暴灵光激活时,你可以选择一个你可以看见的、处于灵光里的生物。该生物必须通过一次力量豁免,豁免失败者速度降为0,且直到你的下一个回合开始,其被如同魔法的冰霜冻住。
\subsection{狂热者 Zealot}一些神灵激励他们的追随者们投身于残酷的战斗狂怒中。这些野蛮人是狂热的战士,他们引导自己的愤怒转化为强大的神赐力量。

在D\&D世界中有许多神祇都会鼓励自己的追随者寻求狂热者道途。最著名的是费伦的Tempus,灰鹰的Hextor和Erythnul。一般来说,那些培养狂热者的神祇往往是偏向于战斗、破坏和暴力的神职。当然,并非所有这样的神都是邪恶的,但只有少数例外。
\header{特性}
\begin{dndtable}[cX]
\textbf{等级} & \textbf{特性} \\ 
3rd & \emph{\specialcell{神性之怒 Divine Fury,\\ 神之勇者 Warrior of the Gods }}\\ 
6th & \emph{狂热专注 Zealous Focus}\\ 
10th & \emph{狂热存在 Zealous Presence }\\ 
14th & \emph{怒不畏死 Rage be yond Death}\\ 
\end{dndtable}
\subsubsection{神性之怒 Divine Fury}你在第3级选择此道途起,你可以引导神性之怒注入武器打击中。若你处于狂暴,你在自己回合中用武器攻击命中的第一个生物将额外受到伤害,数值等于1d6+你野蛮人等级的一半。此伤害的类型为黯蚀或光耀,而你在你得到此特性时选择其类型。
\subsubsection{神之勇者 Warrior of the Gods}第3级时,你的灵魂被无尽的战斗所烙印。若一个法术——例如\emph{死者复活 raise dead}——只具有让你起死回生(但不是变为不死生物)的效果,该施法者在对你施展法术时不需要材料。
\subsubsection{狂热专注 Zealous Focus}第6级起,驱使你愤怒的神力也能庇护你免遭损害。若你在狂暴时未通过一次豁免检定,你可以重新投掷该豁免检定,但你必须接受第二次的检定数值。你每次狂暴只能使用此能力一次。
\subsubsection{狂热存在 Zealous Presence}第10级时,你学会了引导神力,让其他人也被狂热所感染。作为一个附赠动作,你愤怒的怒吼并释放出一道蕴含神圣能量的战吼。选择至多10个距离你60尺以内并且能听到你吼叫的其他生物,直到你的下个回合开始,这些生物在攻击检定和豁免检定中获得优势。

你必须完成一次长休才能再次使用此特性。
\subsubsection{怒不畏死 Rage be yond Death}第14级起,驱使你狂暴的神力允许你忽视致命的打击。

当处于狂暴时,降至0点生命值不会让你昏迷。你仍必须进行死亡豁免检定,而你在0生命值下受到伤害时承受正常的效果。然而,若你因为未通过死亡豁免检定而将死去,则直至狂暴结束前,你不会死亡,届时如果生命值仍为0则你才会死亡。

