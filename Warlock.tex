%!TEX root = Mainbook.tex
\clearpage
\chapter{契术师 Warlock}
\begin{quote}
\emph{你觉得我疯了?我倒觉得明知道在尘世之外,知识和力量触手可及,你们却整天营营苟且于俗事,这才真是疯了呢。——Xarren,阿卡玛之兆景}
\end{quote}

契术师是探密者,契术师是守密人。他们扩张着我们对世界认知的边界,总是寻觅超越他们理解的事物。当贤者和法师们发现确凿的危险预兆,从而终止他们的实验时,契术师们却勇往直前,不计后果。因此,一个契术师总是由天分、好奇和鲁莽混合塑造成的独特存在。许多普通人会把这些特质当成疯狂的典型证据;契术师们则将其视为勇气的象征。

契术师由两个基本要素塑造;这两个基本要素相互协作,铸就契术师生涯的道路。第一个元素是,是什么事件亦或是环境,导致了一名契术师与一个位面存在订立了契约;第二个元素是,契术师所立契的存在,究竟有什么样的天性。和主动接受神祇和神祇理念的牧师不同,一个契术师有可能对宗主毫无认同,也有可能完全相反。

接下来的章节提供了塑造契术师角色的一些方法,帮助你创造复杂有趣的故事和角色扮演的契机。

\section{宗主态度 Patron's Attitude}所有的关系都是相互的,但对于契术师和他们的宗主而言,双方的关系就不一定平等了。契术师对他的宗主所持有的态度,无论是积极的还是消极的,可能会被宗主同等地回报,但也可能契约双方对彼此有完全相反的感情。

当你决定你的契术师角色对宗主抱持的态度时,也要考虑宗主一方是如何看待的。你的宗主会对你做什么?你的宗主是一个朋友和同盟,还是因你威逼利诱下才赐予你力量的敌手?

\begin{dndtable}[cX]
\textbf{d6} & \textbf{态度} \\
1 & \emph{你的宗主世世代代引导、帮助你的家族,并且对你也十分友善。} \\
2 & \emph{每次与你任性宗主的互动都令人惊异,无论是惊喜还是惊恐。} \\
3 & \emph{你的宗主是一位逝去已久的英雄,他将与你立下的契约视作继续影响世界的方法。} \\
4 & \emph{你的宗主极度严格地奉行规矩,但对你抱着尊重的态度。} \\
5 & \emph{你的宗主欺骗你立下契约,并把你当做奴隶。} \\
6 & \emph{你基本上被丢在一边做自己的事,宗主几乎无所作为。有的时候你会担心它出现时将会提出什么要求。} \\
\end{dndtable}

\section{契约特别条款 Special Terms of the Pact}一份契约可能是松散如口头约定,也可能严格如有冗长而详细的条款、附有需求列表的正式合同。契约的条款——契术师为了获取宗主的恩惠,必须去做的事情——总是由宗主来决定。偶尔这些条款里会包含古怪或者完全异想天开的条目,但契术师们会将这些条目与契约中其他条目一样严肃对待。

你的角色是否有一份需要你以不寻常或者琐碎古怪的方式,改变你的行为的契约?即使你的宗主还没有加予你这样的义务,还是不能说以后也仍然不会。

\begin{dndtable}[cX]
\textbf{d6} & \textbf{条款} \\
1 & \emph{当指示下达时,你必须立刻对宗主的一名特定的敌人采取行动。} \\
2 & \emph{你的契约考验着你的意志力;你不能饮酒或者类似的醉人之物。} \\
3 & \emph{每天至少一次,你必须在墙上涂鸦或刻下宗主的名讳或者徽记。} \\
4 & \emph{你偶尔需要进行古怪的仪式来维持契约。} \\
5 & \emph{你不能第二次穿上曾经穿过的外套;你的宗主认为这种可预测性非常无聊。} \\
6 & \emph{当你使用魔能祈唤时,要么高声呼喊你宗主的名字,要么承担它感到不快的风险。} \\
\end{dndtable}

\section{契约印记 Binding Mark}有些宗主有这样的习惯——或者,更通常的情况是,爱好——以某种形式宣称契术师在它们的掌控之下。对那些知道内情的人而言,一个契约印记会很明确的表现出面前这个个体受缚于服务这位宗主的义务。一个契术师可能能通过展示作为契约证据的印记获得一些优势,也可能选择让它隐藏在遮蔽物下(如果可能的话)来避免它带来的难题。

如果你的契术师契约带来一个契约印记的话,你对印记的感受可能取决于你和给予你印记的宗主的关系。这印记是你骄傲的源泉呢,还是你暗中引以为耻的东西?

\begin{dndtable}[cX]
\textbf{d6} & \textbf{印记} \\
1 & \emph{你其中一只眼睛看起来和宗主的眼睛一模一样。} \\
2 & \emph{每次你醒来,你脸上的一个小瑕疵都会改变位置。} \\
3 & \emph{你的外表显示出疾病的症状,但对你并没有实际危害。} \\
4 & \emph{你的舌头颜色很不自然。} \\
5 & \emph{你有一条返祖的尾巴。} \\
6 & \emph{你的鼻子在暗处会发光。} \\
\end{dndtable}

\section{异界宗主 Otherworldly Patrons}从1级起,契术师获得异界宗主职业特性。在玩家手册提供的职业选项之外,契术师可以选择下列的职业选项:天界宗主和咒剑士。

\subsection{天界宗主 Celestial}你的宗主是上层位面的强大存在。你和一位古老的至高天、炽天神侍、麒麟、独角兽,或其他居住于那些永享极乐的位面中的存在立下了契约。你与它们的契约让你体会到裸身沐浴在照亮多元宇宙的圣光中的独特体验。

沐浴在如此力量中,会改变你的行为和信念。你可能发现自己投身于消灭不死生物、抵御魔族、保护无辜者的事业当中。不时地,你的心中会充斥着对你宗主拥有的天国的渴望,期冀于在你的余生中享有的天堂美景。但你知道你现在的任务是栖身于凡世之中,并给世界的黑暗角落带去光明。

\begin{dndtable}[cX]
\textbf{等级} & \textbf{特性} \\
1st & \emph{\specialcell{扩展法术列表 Expanded Spell List,\\额外戏法 Bonus Cantrips,\\治愈之光 Healing Light}} \\
6th & \emph{光耀之魂 Radiant Soul} \\
10th & \emph{天界韧性 Celestial Resilience} \\
14th & \emph{灼光复仇 Searing Vengeance} \\
\end{dndtable}

\subsubsection{扩展法术列表 Expanded Spell List}当你学习契术师法术时,天界宗主允许你从扩展法术列表中选择法术。对你而言,下列法术加入到契术师法术列表中。

\begin{dndtable}[cX]
\textbf{法术等级} & \textbf{法术} \\
1st & \emph{疗伤术cure wounds,曳光弹guiding bolt} \\
2nd & \emph{\specialcell{炽焰法球 Flaming Sphere,\\次级复原术 Lesser Restoration}} \\
3rd & \emph{昼明术 Daylight,回生术 Revivify} \\
4th & \emph{\specialcell{信仰守卫 Guardian of Faith,\\火墙术 Wall of Fire}} \\
5th & \emph{焰击术 Flame Strike,高等复原术 Great Restoration} \\
\end{dndtable}

\subsubsection{额外戏法 Bonues Cantrps}第1级时,你学会了\emph{光亮术 Light}和\emph{圣火术 Sacred Flame}两个戏法。他们对你而言是契术师戏法,但并不计入到你的契术师可知戏法总数中。

\subsubsection{治愈之光 Healing Light}第1级时,你得到了引导天界能量以治疗伤口的能力。你获得了一个d6骰池,用以使用此治愈能力。骰池的总数等于1+你的契术师等级。

以一个附赠动作,你可以消耗骰池,治疗一个你视线中的、距你60尺以内的生物。单独一次能消耗的最大骰子数量等于你的魅力调整值(最小为1)。投掷你消耗的骰子,将骰值相加来决定所能恢复的生命值总量。

当你完成一次长休后,你恢复所有骰池中的骰子。

\subsubsection{光耀之魂 Radiant Soul}第6级起,你所连接的天界宗主允许你成为光耀能量释放的渠道。你获得对光耀伤害的抗性,当你施展一个造成光耀或火焰伤害的法术时,你可以将你的魅力调整值加到法术对其中一个目标造成的一次光耀或火焰伤害骰上。

\subsubsection{天界韧性 Celestial Resilience}第10级起,当你完成一次长休或短休后,你会得到临时生命值。得到的临时生命值等于你的契术师等级+你的魅力调整值。另外,在你完成休息时,你可以选择最多五个视线中的生物。这些生物各自获得等于你的契术师等级的一半+你的魅力调整值的临时生命值。

\subsubsection{灼光复仇 Searing Vengeance}第14级起,你引导的光耀能量能让你抵抗死亡。当你在回合开始需要进行一次死亡豁免检定时,你可以选择在一次光耀能量的爆发中一跃而起。你恢复等于你最大生命值一半的生命值,如果你愿意的话可以选择处于站立状态。距你30尺以内的、你所选择的所有生物受到2d8+你的魅力调整值的光耀伤害,并且目盲到你当前回合的结束。

你必须完成一次长休才能再次使用此特性。


\subsection{咒剑士 The Hexblade}你和一个来自堕影冥界的神秘实体订立了契约,那些由阴影物质雕琢的智能魔法武器隐约显露了其力量。强悍的魔法剑黑刃blackrazor是这些武器中最令人瞩目的,它在无数年间辗转于多元宇宙之中。这些武器背后的暗影之力,能为与其立契的契术师提供力量。许多咒剑契术师模仿那些在堕影冥界中锻造的武器塑造自己的兵刃,而其他人则选择放弃这些武器,转向链接堕影冥界的暗影魔法网络来施展法术。

鉴于人尽皆知是鸦后锻造出了最初的此类武器,许多贤者推测她就是暗影武器的幕后黑手,而这些武器——以及咒剑契术师们——都是她用来玩弄物质位面的是非,以实现她难以揣测的目的的工具。


\begin{dndtable}[cX]
\textbf{等级} & \textbf{特性} \\
1st & \emph{\specialcell{扩展法术列表 Expanded Spell List,\\咒剑巫咒 Hexblade's Curse,\\巫咒战士 Hex Warrior}} \\
6th & \emph{咒缚幽灵 Accursed Specter} \\
10th & \emph{巫咒盔甲 Armor of Hexes} \\
14th & \emph{巫咒大师 Master of Hexes} \\
\end{dndtable}


\subsubsection{扩展法术列表}
\begin{dndtable}[cX]
\textbf{等级} & \textbf{法术} \\
1st & \emph{护盾术 shield,怒火斩 wrathful smite} \\
2nd & \emph{朦胧术 blur,印记 branding smite} \\
3rd & \emph{闪现术 blink,元素武器 elemental weapon} \\
4th & \emph{\specialcell{幻影杀手phantasmal killer,\\惊惧斩staggering smite}} \\
5th & \emph{放逐斩banishing smite,寒冰锥cone of cold} \\
\end{dndtable}

\subsubsection{咒剑诅咒 Hexblade’s Curse}第1级起,你得到了向一个敌人施加恶毒诅咒的能力。作为一个附赠动作,选择一个处于你周围30尺内,你能看见的生物。该目标被诅咒1分钟。该诅咒将会在目标死亡、你死亡、你陷入失能时提前结束。直到诅咒结束,你得到下述的好处:
\begin{itemize}
\item 你对被诅咒目标的伤害检定中得到加值。加值等于你的熟练加值。
\item 你对被诅咒目标的任何攻击检定,若在d20中投出19或20,则形成一次重击。
\item 若被诅咒的目标死亡,你恢复等于你魔导士等级+你的魅力调整值(至少为1)的生命值。
\end{itemize}

你必须完成一次长休或短休才能再次使用此特性。

\subsubsection{巫咒战士 Hex Warrior}在1级时,你获得了有效的战斗训练。你得到对于中甲,盾牌,和军用武器的熟练。此外,你宗主的影响也让你得以通过一把特殊武器来魔法的引导你的意志。当你结束一次长休,你都能够触摸一把你已经熟练的、非双手武器,你可以在攻击和伤害检定中,使用你的魅力调整值,以代替力量或敏捷。这个在增益保留到你再次完成长休。如果你在后续获得锋刃契约,此好处将扩展到你用这个魔契召唤的每一把契约武器上,没有武器类型的限制。


\subsubsection{咒缚幽灵 Accursed Specter}从6级开始,你可以诅咒一个命丧你手者的灵魂,暂时束缚它为你服务。当你杀死一个类人生物,你可以让它的灵魂从尸体出现,成为一个缚灵(specter),其数据见怪物手册。当缚灵出现时,它获得等同于你魔导士等级一半的临时生命。为缚灵的投掷先攻,它有自己的回合、服从你的口头命令,并获得特殊的命中加值,加值等于你的魅力调整值(最少0)。

缚灵一直为你服务直到你的下一个长休结束,然后它消失进入来世。

你必须完成一次长休才能再次使用此特性召唤缚灵。

\subsubsection{巫咒盔甲 Armor of Hexes}第10级时,你的巫咒变得更加强大。若你以咒剑诅咒所指定的目标以一次攻击检定击中你,你可以用你的反应投一次d6。在4或更高的结果下,这次攻击改为对你失手,不论骰值是多少。

\subsubsection{巫咒大师 Master of Hexes}从14级起,你可以转移你的咒剑诅咒,从一个被杀的生物转移到另一个生物。当被你的咒剑诅咒所指定的生物死亡时,假如你并未失能,你可以将诅咒转移到一个距离你30尺内你能看到的其他生物身上。当你以此法转移诅咒,你不再从先前目标的死亡中获得生命值回复。

\section{魔能祈唤 Eldritch Invocations}在2级时,契术师获得魔能祈唤特性。在玩家手册的选项之外,这里是一些魔能祈唤的新可选项。

如果一个魔能祈唤有先决条件,你必须满足先决条件才能学习魔能祈唤,你可以在满足先决的同时学习这个魔能祈唤。对等级的要求意为对你契术师职业等级的要求。

\subsubsection{月之仪态 Aspect of the Moon}
\emph{先决:书卷魔契特性}

你不再需要睡眠,同样不能以任何手段强制入梦。为了获得长休的益处,你可以花费8小时执行轻度的工作,例如阅读你的影之书和放哨。

\subsubsection{飞蝇斗篷 Clocks of Flies}
\emph{先决:等级5}

以一个附赠动作,你周身环绕起外观如同纷飞蚊蝇般的魔法灵光。灵光从你四周散发5尺,但不会穿过全掩蔽。灵光持续到你失能或以一个附赠动作解消它。

灵光给予你在魅力(威吓)检定的优势,但你在所有其他魅力检定上遭受劣势。任何在灵光内开始回合的生物,受到等于你魅力调整值的毒素伤害(最小为0)。

当你使用此魔能祈唤后,在完成一次短休或长休前,你无法再次使用这祈唤。

\subsubsection{魔能斩击 Eldritch Smite}
\emph{先决:等级5,锋刃魔契特性}

每回合一次,当你以契约武器命中一个生物,你可以消耗一个契术师法术位,对目标造成额外1d8+每法术位等级1d8的力场伤害,如果目标体型是巨型或更小,你可以将它击倒。

\subsubsection{幽魂凝视 Ghostly Gaze}
\emph{先决:等级7}

以一个动作,你得到看穿30尺范围的固态物体的能力。这种特殊的视觉持续1分钟,或直到你的专注终止(如同你专注于一个法术)。在此期间,你观察物体所看到的是影子般的、透明的画面。

当你使用此魔能祈唤后,在完成一次短休或长休前,你无法再次使用这祈唤。

\subsubsection{深海赠礼 Gift of the Depth}
\emph{先决:等级5}

你可以在水下呼吸,并且你得到等于你步行速度的游泳速度。

你可以施展一次\emph{水下呼吸water breathing}法术,而无需消耗法术位。当你结束一次长休后,你重新获得此能力。

\subsubsection{永生者赠礼 Gift of the Ever-Living Ones}
\emph{先决:锁链默契}

当你的魔宠距离你100尺内时,若你恢复生命值,所有决定你恢复生命值点数的骰子,都被视为骰出了最大值。

\subsubsection{哈达之攫 Grasp to Hadar}
\emph{先决:魔能爆eldritch blast戏法}

每回合一次,当你以魔能爆击中一个生物时,你可以让此生物直线向你移动10尺。

\subsubsection{进阶契约武器 Improved Pact Weapon}
\emph{先决:锋刃魔契特性}

你可以使用任何锋刃魔契职业特性召唤出来的武器作为契术师法术的法器。

除此之外,武器在攻击骰和伤害骰上获得+1加值,除非它是一件已经有攻击骰和伤害骰加值的魔法武器。

最后,你塑造的武器可以是短弓、长弓、轻弩或者重弩。

\subsubsection{怠惰之枪 }Lance of Lethargy
\emph{先决:魔能爆戏法}

每回合一次,当你以魔能爆击中一个生物时,你可以降低目标10尺的速度,直到你的下个回合结束。

\subsubsection{癫狂巫咒 Maddening Hex}
\emph{先决:等级5,法术脆弱诅咒 hex或进行诅咒的契术师特性}

以一个附赠动作,对你以脆弱诅咒或咒剑诅咒、凶兆符记等契术师特性所诅咒的目标,你向他施加一种精神干扰。当你如此做时,你对诅咒目标和所有距他5尺内的、你所选择的生物造成精神伤害。精神伤害值等于你的魅力调整值(最低为1)。为了使用此祈唤,你必须能看到诅咒目标,且他必须距你30尺以内。

\subsubsection{残酷巫咒 Relentless Hex}
\emph{先决:等级7,脆弱诅咒hex法术或进行诅咒的契术师特性}

你的诅咒为你和你的目标间建立了短暂的链接。以一个附赠动作,你可以魔法性质地传送到诅咒目标5尺以内的、你能看到的未占据空间,传送距离最多30尺。为了实现此方式的传送,你必须可以看到诅咒目标。

\subsubsection{暗影寿衣 Shroud of Shadow}
\emph{先决:等级15}

你可以施展\emph{隐身 invisibility}法术,而无需消耗法术位。

\subsubsection{莱维斯图斯之墓 Tomb of Levistus}
\emph{先决:等级5}

当你受到伤害时,你可以以一个反应将自己封锁在冰墓中,冰墓会在你的下一个回合结束时融解。你得到每契术师等级10点的临时生命值,此生命值会尽可能优先地抵消伤害。当你受到伤害之后,你立即得到对火焰伤害的易伤,速度降低至0,并且陷入失能状态。这些效果,以及剩余的临时生命值,会在冰墓融解时终止。

当你使用此魔能祈唤后,在完成一次短休或长休前,你无法再次使用这祈唤。

\subsubsection{诡术师的逃脱术 Trickster's Excape}
\emph{先决:等级7}

你可以对自己施展一次\emph{自由行动 freedom of movement}法术,而无需消耗法术位。当你结束一次长休后,你重新获得此能力。
