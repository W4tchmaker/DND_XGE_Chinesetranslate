%!TEX root = Mainbook.tex
\clearpage
\chapter{牧师 Cleric}
\begin{quote}
\emph{成为牧师便是成为神之使者。神授之力固然强大,但伴随的责任更重。——大主教 Riggby}
\end{quote}

在世间,几乎所有信奉神明的人都没有直接接触过神明,因此,他们中的很多人都不知道作为牧师是什么感受——牧师不只是虔诚的崇拜者,而且被赐予了神的力量。

有一个问题被争议已久——凡人成为牧师,是因为他对某个神的虔诚信仰,从而获得了神的青睐?还是神明发现了他的潜能并且感召他来为自己服务?说到底,也许这个答案并不重要。不论是如何成为牧师,这个世界都需要他们,如同他们和神明需要彼此。

如果你要扮演一个牧师角色,以下内容会提供一些方法来完善他的过去和性格。
\section{神庙 Temple}

大多数牧师按部就班地从成为一名教士开始他们侍奉神的生活,之后,他们意识到自己被神祝福,拥有成为一名牧师的资格。为了做好应对新职责的准备,候选牧师们通常会接受来自神庙或者其他侍神之地的牧师的教导。

一些神庙与世隔绝以求可以让人们专注于祈祷,另一些则大开方便之门,服务,治疗民众。你曾呆过的神庙有什么特色?

\begin{dndtable}[cX]
\textbf{d6} & \textbf{神庙} \\
1 & \emph{你的神庙据说是你侍奉的神的神庙中现存最古老的。 }\\ 
2 & \emph{几名同盟的神明的侍奉者们都在此接受神明的引导。 }\\ 
3 & \emph{你来自一个以酿酒闻名的神庙。有的人说你闻起来就像是那里产的酒。 }\\
4 & \emph{你的神庙是一座堡垒,一个战斗祭祀的训练场。}\\ 
5 & \emph{你的神庙和平而简朴,充满了菜园和朴素的牧师。 }\\ 
6 & \emph{你来自外层位面的神庙。}\\ 
\end{dndtable}

\section{信物 Keepsake}
许多牧师的个人装备中有一些象征着他们的信仰,提醒他们的誓言,或者其他的,可以帮助他们不走上邪路的物品。即使这样的物品没有神圣力量,它也因其象征意义而对他的主人无比重要。

\begin{dndtable}[cX]
\textbf{d6} & \textbf{信物} \\
1 & \emph{圣徒的手指骨 }\\ 
2 & \emph{金属装订的,讲述如何狩猎和毁灭地狱生物的书 }\\ 
3 & \emph{猪哨子,可以让你回忆起你谦虚而被敬爱的老师 }\\
4 & \emph{独角兽尾巴毛辫}\\ 
5 & \emph{讲述如何最好地消灭亡灵巫师的书 }\\ 
6 & \emph{据说被神祝福的符文石}\\ 
\end{dndtable}

\section{秘密 Secret}

没有那个凡物的灵魂完全摆脱他人的影响或怀疑,即使是牧师也要和背离神的教导的黑暗欲望以及禁忌诱惑斗争。

如果你从未思考过你角色的这一面,看一看这个提供一些可能的表格,或者用它们来获取灵感。你深沉,黑暗的秘密可能和你做过,或是正在做的事情有关,亦或它就根植于你对世界的看法和你在中扮演的角色。

\begin{dndtable}[cX]
\textbf{d6} & \textbf{秘密} \\
1 & \emph{一个小恶魔给你提供“忠告”,你试图无视它,但有时它的建议的确有用。}\\ 
2 & \emph{你相信,归根结底,神明只是有超强力量的凡物。 }\\ 
3 & \emph{你承认神的力量,但是你认为大多数事情都纯属偶然。 }\\
4 & \emph{即使你可以使用神术,你从未真正感到自己内心圣洁本质的存在。}\\ 
5 & \emph{神明因为你某些未知的错误而驱逐你的噩梦折磨着你。 }\\ 
6 & \emph{在绝望时,你认为你不过是诸神的玩物,你憎恨他们的高高在上。}\\ 
\end{dndtable}

\setthemecolor[PhbTan]

\begin{paperbox}{\textbf{侍奉众神,哲学观念,原理}}

一名典型的牧师是某一个特定神明的侍者,他会选择一个属于神明的神圣领域。他的神术源于神明或是神明的神国,同时,他总会携带象征着这位神明的圣徽。

一些牧师,特别是艾伯伦的牧师,侍奉整个神系而不是某个神明。在一些模组中,一名牧师可能转而侍奉宇宙原力,例如生命或死亡,或是哲学观念或概念,例如爱情,和平,或是九大阵营之一。在DMG第一章的“世界上的神”部分中提供了这些选项。

和你的DM聊一聊模组中的这些信仰选项,不论是神,众神,哲学观念还是宇宙原力。不论你最终侍奉什么生物或是事物,为其选择一个适合的神圣领域。如果你侍奉的存在没有圣徽,和你的DM一同设计一个。

牧师的职业特性通常与神有关。如果你投身于众神,哲学观念或是宇宙原力,你的职业特性如同你选择的那样起效。将提及神的部分看作是提及你侍奉的,赐予你神术的神圣存在。
\end{paperbox}

\setthemecolor[Phbclasstable]

\section{神圣领域 Divine Domains}

第1级时,牧师将获得神圣领域特性,下列选项给出了在玩家手册提供的神圣领域以外的额外神圣领域选项:锻造领域和墓冢领域。

\subsection{锻造领域 Forge Domain}

锻造之神是铁匠的守护神,从村里的做马蹄铁和犁的小铁匠,到制作过击败了恶魔领主的钻石尖秘银箭的强大精灵工匠。锻造之神教导人们,锲而不舍,金石可镂。这些神明的牧师寻找因黑暗力量失落的物品,解放被兽人占领的矿山,发掘制造强力魔法物品必须的珍惜而奇妙的物品。他们为自己的工作自豪并且乐于制造和使用重甲和强力的武器保护自己。锻造之神包括贡德,李奥克斯,昂那塔,莫拉丁,赫淮斯托斯和吉欧布尼乌。

\header{锻造领域特性}
\begin{dndtable}[cX]
\textbf{等级} & \textbf{特性} \\
1 & \emph{\specialcell{领域法术 Domain Spells,\\附赠熟练项 Bonus Proficiencies,\\锻造祝福 Blessing of the Forge}} \\
2 & \emph{\specialcell{引导神力:铁匠祝福 \\Channel Divinity: Artisan's Blessing}} \\
6 & \emph{锻造之魂 Soul of the Forge} \\
8 & \emph{锻造之魂 Soul of the Forge(1d8)} \\
14 & \emph{锻造之魂 Soul of the Forge(2d8)} \\
17 & \emph{火与钢的圣徒 Saint of Forge and Fire} \\
\end{dndtable}

\subsubsection{领域法术 Domain Spells}

你依据等级获得在锻造领域法术一表中列出的法术,查阅神圣领域职业特性,以获悉这些法术如何生效。

\begin{dndtable}[cX]
\textbf{等级} & \textbf{法术} \\
1 & \emph{鉴定术 identify,炽焰斩 searing smite} \\
3 & \emph{灼热金属 heat metal,魔化武器 magic weapon} \\
5 & \emph{\specialcell{元素武器 elemental weapon,\\防护能量伤害 protection from energy}} \\
7 & \emph{鬼斧神工 fabricate,火墙术 wall of fire} \\
9 & \emph{活化物件 animate objects,造物术 creation} \\
\end{dndtable}

\subsubsection{附赠熟练项 Bonus Proficiencies}

在你1级选择锻造领域时,你获得重甲和铁匠工具熟练项。

\subsubsection{锻造祝福 Blessing of the Forge}
第1级时,你获得将魔法灌输到武器或护甲中的能力。在长休结束时,你可以触摸一件非魔法护甲或者简易或军用武器,直到你下一次长休或者你死亡,该装备变成一见魔法物品,若是护甲,其AC获得+1加值;若是武器,其攻击骰和伤害骰获得+1加值。

你必须完成一次长休才能再次使用此特性。

\subsubsection{引导神力:铁匠祝福 Channel Divinity: Artisan's Blessing}

从2级起,你可以以引导神力创造简易物品。

你可以举行一个一小时的仪式,以创造包含有金属的非魔法物品:一件护甲,一把简易或军用武器,十份弹药,工具或是其他物品(参考PHB的物品一章)。在这一小时结束时,被创造的物品会出现在你选择的,一个五尺内且未被占据的空间内。

你创造的物品的价值必须低于100gp。作为仪式的一部分,你必须提供等同于物品价值的金属物品,比如钱币。在仪式的结束,这些物品会不可逆地变成你创造的物品,魔法会形成非金属的部分。

仪式可以创造包括金属的非魔法物品的复制品,例如钥匙,但你在仪式过程中必须拥有原件。

\subsubsection{锻造之魂 Soul of the Forge}

第6级起,你对锻造的掌控给你带来额外的增益。
\begin{itemize}
\item 你对火焰伤害具有抗性。
\item 当你身着重甲,你的AC获得+1加值。
\end{itemize}
\subsubsection{神圣打击 Divine Strike}
第8级时,你获得将你锻造的火热力量灌输到武器打击中的能力。由你发动的武器攻击命中生物时,你可以对目标额外造成1d8的火焰伤害,该能力在你的每回合可发动一次。14级时,额外伤害增为2d8。

\subsubsection{火与钢的圣徒 Saint of Forge and Fire}
第17级时,你和火与钢的联系更加紧密。
\begin{itemize}
\item 你免疫火焰伤害。
\item 当你身着重甲,你对源于非魔法攻击的钝击,穿刺,挥砍伤害具有抗性。
\end{itemize}

\subsection{坟墓领域 Grave Domain}
死神俯瞰生死之线,对他们来说,死亡和死后生活是多元宇宙的基础组成,玷污死者的宁静不可饶恕。死神包括Kelemvor,Wee Jas,不朽评议会的Undying Court,Hades,Anubis和Osiris。他们的追随者让彷徨的灵魂安息,摧毁不死生物,安抚垂死之人。他们的神术可以暂时让人远离死亡,特别是那些有重要的事要做的人。这是延迟死亡而非否定死亡,死亡终究到来。
\begin{dndtable}[cX]
\textbf{牧师} & \textbf{特性} \\
1st & \emph{\specialcell{领域法术 Domain Spells,\\ 鉴死之环 Circle of Mortality,\\坟墓之眼 Eyes of the Grave}} \\
2nd & \emph{\specialcell{引导神力:往墓之途 \\Channel Divinity: Path to the Grave}} \\
6th & \emph{死卫 Sentinel at Death's Door} \\
8th & \emph{强力施法 Potent Spellcasting} \\
17th & \emph{守魂者 Keeper of Souls} \\
\end{dndtable}

\subsection{领域法术 Domain Spells}
你依据等级获得在墓冢领域法术一表中列出的法术,查阅神圣领域职业特性,以获悉这些法术如何生效。
\begin{dndtable}[cX]
\textbf{等级} & \textbf{法术} \\ 
1 & \emph{灾祸术 bane,虚假生命 false life} \\
1 & \emph{\specialcell{遗体防腐 gentle repose,\\ 衰弱射线 ray of enfeeblement}} \\
1 & \emph{回生术 revivify, 吸血鬼之触 vampiric touch} \\
1 & \emph{枯萎术 blight, 防死结界 death ward} \\
1 & \emph{防活物护罩 antilife shell, 死者复活 aise dead} \\
\end{dndtable}

\subsubsection{死之环 Circle of Mortality}
第1级时,你获得操控生死之线的能力,当你因使用法术为生命值为0的生物恢复生命而投掷骰子时,你的每个骰子都可以直接取满。

此外,你获得戏法\emph{维生术 spare the dying },且不计入你已知的牧师戏法。对你来说,此法术射程为30尺,并且你可以用附赠动作施展该法术。

\subsubsection{坟墓之眼 Eyes of the Grave}
第1级时,你获得偶尔可以意识到不死生物存在的能力,因为他们的存在本身就是对生命循环的亵渎。作为一个动作,你可以伸张意识来魔法性地辨识不死生物。直到你的下回合结束,你可以定位60尺内的没有处于全掩护之下或被预言魔法保护的不死生物,这种感知不会告诉你生物的能力和身份。

你可以使用此能力的次数等同于你的感知调整值(至少为1),完成一次长休后,你将恢复所有的坟墓之眼使用次数。

\subsubsection{引导神力:往墓之途 Channel Divinity: Path to the Grave}
第2级起,你可以使用\emph{引导神力}来消亡敌人的生命力。
作为一个动作,你诅咒30尺内的一个可视生物直至你的下回合结束。当被诅咒的生物下一次被你或你的盟友的攻击命中时,他对此次攻击的所有伤害具有易伤,之后诅咒终止。

\subsubsection{死卫 Sentinel at Death's Door}
第6级时,你可以阻碍死亡的进程。作为一个反应,当你或你30尺内的可见盟友被重击,你可以将此重击变为普通命中,并取消所有重击引起的效应。
你可以使用此能力的次数等同于你的感知调整值(至少一次),完成一次长休后,你将恢复所有的死卫次数。

\subsubsection{强力施法 Potent Spellcasting}
第8级起,你当你施展牧师戏法,你可以额外造成等同于你感知调整值的伤害。

\subsubsection{守魂者 Keeper of Souls}
第17级起,你可以抓住游魂的一丝生命力来治愈生命。当你60尺内可见的敌人死亡,你或者你选择的60尺内的一个生物恢复等同于死亡敌人生命骰数量的生命。你只有不处于失能状态时才可使用此特性。你必须等你的下回合开始才能再次使用此特性。