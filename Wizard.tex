%!TEX root = Mainbook.tex
\clearpage
\chapter{法师 Wizard}
\begin{quote}
\emph{巫艺需要理解,了解魔法怎样而又为何运作。我们努力增进对其的理解,并已然铸就数个世纪以来文明的超卓进步。——Gimble,幻术师}
\end{quote}

世间仅有少数人能施行魔法,而在他们之中,法师站在此技艺之巅。纵然是他们中最差劲的,也能够操纵魔力并嘲笑大自然的法则。而他们中学识最为渊博者,其法术能让整个世界为之战栗。法师为他们的技艺付出最昂贵的代价便是时光。学习、授课与实验耗费数年的光阴,以求驾驭魔法并将之牢记于心。对于冒险的法师与其他向往着魔法的至高境界的施法者来说,学习永不终结,对力量与知识的渴求亦如是。

假若你正在扮演一个法师,利用好这个机会,使得你的角色不仅仅是一个老套的念咒机。使用下面的建议添加一些有趣的细节,诠释你的法师与世界的交互方式。

\section{法术书 Spellbook}你的法师角色最为珍贵的财富——你的法术书,可能看起来平淡无奇,封面并未暗示着里面的内容。或者你可能布置一些特别之处,就如同许多法师那样,创造一个与众不同的法术书。假如你还没有这样的玩意儿,你的目标之一可能便是找寻一个让你不同于他人的法术书——因其独特的外观或制造手段。

\begin{dndtable}[cX]
\textbf{d6} & \textbf{法术书} \\
1 & \emph{铁铸卷册,酸蚀为文。} \\
2 & \emph{一个破损的卷宗,写满了仅有你能识别的象形文字。} \\
3 & \emph{写满咒文的长带的皮革,为便于携带而缠于法杖之上。} \\
4 & \emph{刻有符文的石头,存于布袋之中。} \\
5 & \emph{一本被龙炎所焚坏的书,裸露出其中的法术笔迹。} \\
6 & \emph{由完全漆黑的纸面组成的卷册,笔迹仅于昏暗与黑暗中可见。} \\
\end{dndtable}


\section{野心 Ambition}鲜有法师扛起魔法研习的重担而不抱有个人目的。许多法师将魔法作为获得切实利益的工具,亦或是追求物质,亦或是渴求地位,亦或是为了自己,亦或是为了同伴。对于另外一些法师,魔法的理论本身极富吸引力,推动这些法师们追逐知识,以支撑奥法的新论或验证旧说。

除了那些显而易见的理由,为什么你的法师角色学习魔法?你想达到怎么样的成就?如果你没有仔细思量过这些问题,你现在就可以这样做,而你给出的答案很可能会影响你未来的路。

\begin{dndtable}[cX]
\textbf{d6} & \textbf{野心} \\
1 & \emph{你将证明,诸神并未如人们所想象得那般强大。} \\
2 & \emph{你学海的终点乃是不朽。} \\
3 & \emph{假若你能充分了解魔法,你能够解锁其全部的威能以开创一个平等的新纪元。} \\
4 & \emph{魔法是危险的工具,你以之保卫你所爱。} \\
5 & \emph{奥术的威能必须从滥用者手中夺去。} \\
6 & \emph{你将成为,世上数代人所能见过的,最伟大的法师。} \\
\end{dndtable}

\section{怪癖 Eccentricity}夜以继日孤苦的研习,会对任何人的社交能力造成负面影响。纵然是一开始就不同于凡人的法师,亦不例外。一两个怪癖未必是缺点,其提供趣味或是成为多种多样的标签。

假如你的角色拥有一个怪癖,它是身体上的抽搐亦或是心灵上的?你是因为它才在某些圈子里变得众所皆知的吗?你是否为克服之而斗争,亦或是拥抱了那使你名声更盛的戏言。

\begin{dndtable}[cX]
\textbf{d6} & \textbf{怪癖} \\
1 & \emph{你无止地跺脚并时常惹恼周边的人。} \\
2 & \emph{你记忆力超乎凡人,但你可以毫不顾忌地假装心不在焉——如果这符合你的目的的话。} \\
3 & \emph{你从不在看看天花板上悬挂什么之前就走进一个房间。} \\
4 & \emph{你最为珍爱的财富是一个放在药瓶里的死虫子。} \\
5 & \emph{当你希望他人离开你时,你便开始自顾自地说话。} \\
6 & \emph{你的时尚感和妆容,或更多缺乏的东西,时常让人误以为你是个乞丐。} \\
\end{dndtable}

\section{奥术传承  Arcane Traditions}
从2级起,法师获得奥术传承职业特性。在玩家手册提供的职业选项之外,契术师可以选择下列的职业选项:战法师。

\subsection{战法师}许多奥术学院专门培养用于战争的法师。战争之传承精妙地融合了塑能与防护的机理,而非专精于独一学派。它既授予施法者增进法术威力的技巧,同时也为法师提供了改良防御能力的手法。这个奥术传承的追随者被称为战争法师。他们将他们的魔法视为武器和盔甲,比任何一块钢铁都更为坚韧的材料。

战争法师在战斗中行动迅速,用他们的法术夺取对战局的战术控制。他们的法术迅猛非常,而他们的防守技巧挫败了对手的反击势头。战争法师也善于导引其他施法者的魔法能量反过来攻击他们。

于恢弘的大战中,战争法师时常与塑能师、防护师和其他类型的法师相共事。塑能师最常嘲笑战法师分心于攻守之间,一位战法师的典型回应是:“死人可丢不了火球。”

\begin{dndtable}[cX]
\textbf{等级} & \textbf{特性} \\
2nd & \emph{\specialcell{奥术偏斜 Arcane Deflection,\\战术思维 Tactical Wit}}\\
6th & \emph{力量涌动 Power Surge}\\
10th & \emph{耐久魔法 Durable Magic}\\
14th & \emph{偏斜罩幕 Deflecting Shroud}\\
\end{dndtable}

\subsubsection{奥术偏斜 Arcane Deflection}第2级时,你学会了编织魔法以更强韧的面对伤害。当你被一次攻击击中,或未通过一次豁免时,你可以使用你的反应来在对抗那次攻击的AC上获得+2加值,或在那次豁免检定中得到+4加值。

当你使用此特性时,你不能施展戏法之外的法术,直到你下个回合结束。

\subsubsection{战术思维 Tactical Wit}第2级起,你对于战略格局的急速把控让你能在战斗中快速的反应。你在先攻检定中获得等于你智力调整值的加值。

\subsubsection{力量涌动 Power Surge}第6级起,你可以将魔法能量存于体内并于之后增幅你危险异常的法术。在这一存储模式中,这一能量被称为力量涌动。

你不能储存超过你智力调整值(至少为1)的力量涌动。当你完成一次长休,你的力量涌动数重置为1.当你成功用解除魔法或反制法术终结一个法术时,你获得1点力量涌动,因为你从你挫败的法术中窃得魔力。当你在没有力量涌动次数时完成一次短休,你获得1点力量涌动。

每回合一次,当你用一个法师法术对一生物或物体造成伤害时,你可以消耗1点力量涌动来对目标额外造成力场伤害。额外伤害值等于你的法师等级的一半。

\subsubsection{耐久魔法 Durable Magic}第10级起,你引导的魔法也能阻绝伤害。当你于一个法术上维持专注时,你在AC和所有豁免上获得+2加值。

\subsubsection{偏斜罩幕 Deflecting Shroud}第14级时,你的奥术偏斜被灌注了致命的魔法。当你使用你的奥术偏斜特性时,你可以让魔法能量从你身上涌出。至多三个60尺内你能看见的目标受到半数于你法师等级的力场伤害。