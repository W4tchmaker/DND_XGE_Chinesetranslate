%!TEX root = Mainbook.tex
\clearpage
\chapter{战士 Fighter}
\begin{quote}
\emph{聊完喊我。—— Tordek}
\end{quote}

在D\&D世界里所有的冒险者中,战士们之间的分歧可能最大。一方面,没有两个战士会用同样方式来体现职业特性,像他们的武器,铠甲和战斗方式都大相径庭。另一方面,无论他们使用何种武器,又以何种方式战斗,每一个战士心中都有相同的信念:伤害别人好过被别人伤害。

虽然有些战士去冒险是要赌上的是荣誉和财宝,但也有些是为他人的利益着想。他们看重社会、村镇和团体的安定幸福甚于自己的安危。就算金山就在眼前,对大多数战士而言真正的奖励也是送他们的敌人上路。

接下来提供的一些选项能让你的战士角色更深邃,带有更强烈的个人特征。

\section{纹章标志 Heraldic}

战士的战斗需要理由。有些战士会为了王国与怪物而战斗,也有些只是为了寻求个人的荣耀。无论是那种情况,一个战士通常会需要一个标志来彰显原因,你既可以从一个王国或者贵族那里获得封赐,也可以随意的率性而为。

你的角色可能与某个组织或某种理念相连,又或者因为某些愿意踏上旅途。如果你对这些都不满意,你也可以用象征你的本性的纹章,不妨用它诉说你在为什么要来这世上走一遭。

\begin{dndtable}[cX]
\textbf{d6} & \textbf{标志} \\
1 & \emph{一条伫立于绿野上的狂躁金龙,象征着勇猛和对财富的追求} \\
2 & \emph{于云端紧握闪电的风暴巨人之拳,象征着愤怒与力量} \\
3 & \emph{在城堡门前交叉的绿色对剑,象征着保家或卫国} \\
4 & \emph{一颗被匕首刺穿的颅骨,象征着你为敌人带来的覆灭} \\
5 & \emph{火环里的凤凰,象征着不屈的灵魂} \\
6 & \emph{在黑幕中从卧刃落下的三滴血,象征着三名你发誓要铲除的敌人} \\
\end{dndtable}


\section{导师 Instructor}
有些战士天生就是能从战场上存活下来的斗士。还有些年轻时会在军队或其他军事组织里,接受其他导师的教导,以磨炼基础的战斗技巧。

三分之一行伍出身的战士都有过被经验丰富的教官一对一指导练习的经历。根据学生的背景,导师可能曾是,或现在仍是某个专业领域里的佼佼者。

如果你判断你的角色有个私人导师,他又有什么独到之处?你是会模仿你的导师的战斗方式,还是会为了个人目标加以扬弃呢?

\header{导师}
\begin{dndtable}[cX]
\textbf{d6} & \textbf{导师} \\
1 & \emph{角斗士 你的导师曾是为了自由而战的奴隶,或者他为了金钱和名望而甘愿选择角斗士的道路。} \\
2 & \emph{军人 你的导师在一队士兵中工作且了解团队合作。} \\
3 & \emph{城市守卫 群体管控和维护治安是你的导师的拿手好戏。} \\
4 & \emph{部落勇士 你的导师成长在一个赌命干架是家常便饭的部落。} \\
5 & \emph{街头打手 你的导师擅长街头斗殴,近身搏斗的技术安静而高效。} \\
6 & \emph{武器大师 你的导师帮助你高效使用你选择的武器,直至人剑合一。} \\
\end{dndtable}


\section{代表风格 Signature Style}
很多战士磨练和完善的战斗技巧和风格来让他们和其他战士比较起来与众不同。这或许是天性使然,但也未必如此,一个人的战斗风格未必一定代表了他的三观,毕竟这是在生死关头。

你的战斗风格反应了你的内在或外在吗?又或者拔出武器的时候,内心的某个角落随之翻转?

\header{鲜明风格}
\begin{dndtable}[cX]
\textbf{d6} & \textbf{风格} \\
1 & \emph{简练 你的行动精准优雅,收放自如,从没有多余的动作。} \\
2 & \emph{野蛮 你的攻击犹如横冲直撞的锤子,让别人粉身碎骨血沫横飞。} \\
3 & \emph{狡诈 你能够在瞬间做出攻击,喜欢用小把戏争抢上风。} \\
4 & \emph{从容 你很少流汗,在战斗中的表情也宁静从容。} \\
5 & \emph{活力 你在战斗中灵魂高涨地又吼又笑,手握武器,面对强敌让你无比愉悦。} \\
6 & \emph{阴险 你战斗时眉头紧锁,面露讥笑,你尤其享受羞辱你所击败的对手的过程。} \\
\end{dndtable}


\section{武术范型 Martial Archetypes}第3级时,战士将获得其战士泛型特性,下列选项给出了在玩家手册提供的泛型以外的额外泛型选项:魔射手、骑兵和武士。

\subsection{魔射手 Arcane Archer}
魔射手学会了精灵的独有方法,将魔法编织入攻击中,以产生超自然的效果。在精灵中,魔射手也是精锐之师。这些射手们守护着精灵领地的边境,用锐利的目光审视着路过的人,并用灌注魔法的箭矢在怪物和入侵者能侵害到精灵定居点前击败它们。几个世纪以来,这些精灵弓箭手的技巧也被其他种族的成员所学会,他们也学会了这种奥术天分与射术相结合的技巧。

\begin{dndtable}[cX]
\textbf{等级} & \textbf{特性} \\ 
3rd & \emph{\specialcell{魔射手学识 Arcane Archer Lore, \\奥术射击 Arcane Shot(2项)}} \\
7th & \emph{\specialcell{魔法箭矢 Magic Arrow,\\曲线射击 Curving Shot,\\奥术射击 Arcane Shot(3项)}} \\
10th & \emph{奥术射击 Arcane Shot(5项)} \\
15th & \emph{常备射击 Ever-Ready shot} \\
18th & \emph{奥术射击 Arcane Shot(5项)} \\
\end{dndtable}

\subsubsection{魔射手学识 Arcane Archer Lore}
第3级时,你学会了魔法理论或者自然奥秘,这是研究精灵军事的传统艺能。你获得奥术或者自然的熟练项,并选择\emph{魔法伎俩 Prestidigitation}或者\emph{德鲁伊伎俩 Druidcraft}两种戏法之一进行学习。

\subsubsection{奥术射击 Arcane Shot}
在3级时,你学会了在射击时施放些特殊的魔法效果。获得这个特性时,你学会两种由你选择的奥术射击选项(见奥术射击选项栏)。

每回合,当你以一个攻击动作的一部分来用短弓或者长弓射出箭矢时,你可以为你的箭矢挑选一个你有的奥术射击项。你可以在魔法箭击中一名角色时决定是否使用该选项,除非该选项无需攻击骰。你有两次使用该特性的机会,完成一次短休或长休后,你将恢复所有的奥术射击次数。

第7级,第10级,第15级和第18级时,你都将获得一个额外的奥术射击选项,所有奥术射击选项的威力,都将在你18级时得到提升。

\subsubsection{魔法箭矢 Magic Arrow}
第7级时,你学会了在箭矢中灌注魔法。当你用短弓或者长弓射出普通箭矢时,你都可以将其转变为魔法箭矢以克服对非魔法攻击和伤害的抗性和免疫。箭矢上的魔法会起在击中或射失目标后立刻消失。

\subsubsection{曲线射击 Curving Shot}
第7级时,你学会了如何引导偏离的箭矢到新的目标上。当你因魔法箭矢未命中时,你能够以一个附赠动作来指定一个距原目标60尺内的不同目标以重投攻击骰。

\subsubsection{常备射击 Ever-Ready shot}第15级时,无论战斗何时开始你都有魔法箭矢可用。如果你投先攻骰时没有任何奥术射击使用次数,你恢复一次使用机会。

\subsubsection{奥术射击选项 Arcane Shot Options}奥术射击特性允许你在对应等级选择选项。以下选项以字母顺序排列,而且每一个选项都对应一个魔法学派。
如果选项要求豁免骰,你的奥术射击DC等同于8+你的熟练调整值+你的智力调整值。

\subparagraph{放逐箭 Banishing Arrow}你尝试用你的防护魔法来暂时放逐你的目标到一片属于精灵沃野的无害区域。被这发箭击中的角色必须进行一次魅力豁免,豁免失败者将被放逐。当被这种方式放逐时,目标速度降为0,变为失能状态。在他的回合结束时,目标在他被放逐前的位置出现。如果该位置被占据,则他会在最近的未被占据空间出现。

当你达到该职业等级的18级时,目标被箭击中时还会受到2d6的力场伤害。

\subparagraph{欺诈箭 Beguiling Arrow}你的附魔魔法会让箭暂时欺骗目标。被这发箭击中的角色会额外受到2d6的精神伤害。目标必须进行一次感知豁免,豁免失败者则其在你的下一个回合开始前,被由你选择的目标30尺以内的盟友魅惑。而目标被你选择的盟友攻击或受伤时,又或被你选择的盟友强迫他进行一次豁免检定时,该效应中止。

当你达到该职业等级的18级时,该精神伤害增为4d6。

\subparagraph{爆裂箭 Bursting Arrow}你将从塑能魔法中汲取的力场能量注入到箭中,这股能量会在你攻击后爆炸。箭击中目标生物后,其10尺内所有的生物都会立刻受到2d6的力场伤害。

当你达到该职业等级的18级时,该力场伤害增为4d6。

\subparagraph{凋零箭 Enfeebling Arrow}你在你的箭上施展死灵魔法。被这发箭击中的生物会额外受到2d6的黯蚀伤害。目标必须进行一次体质豁免,豁免失败者则其武器攻击造成的伤害在你的下一个回合开始之前减半。

当你达到该职业等级的18级时,该黯蚀伤害增为4d6。

\subparagraph{缠绕箭 Grasping Arrow}当这次攻击击中目标时,咒法魔法会创造一团紧抓着的毒性荆棘来困住目标。被这发箭击中的生物会额外受到2d6的毒性伤害,他的速度减少10尺,当他在每个回合第一次不使用传送而移动1尺以上的距离时他受到2d6的挥砍伤害。这个目标或其他能触碰到荆棘的生物可以使用动作,进行一次对抗你奥数射击DC的力量(运动)检定以消去荆棘。否则,荆棘会在1分钟后,或者你再次使用该选项后结束。

当你达到该职业等级的18级时,该毒性和挥砍伤害增为4d6。

\subparagraph{穿透箭 Piercing Arrow}你使用你的变化系魔法使你的箭变得虚无缥缈。当你使用该选项时,你不为该次攻击投攻击骰,这只箭会在消失前划出一条一尺宽30尺长的直线取而代之。这只箭将无害地穿过物品且无视遮蔽。每一个在这条直线上的生物都必须进行一次对抗你奥术射击DC的敏捷检定。豁免失败者将会受到伤害,该伤害等同被这只箭击中的伤害额外附加1d6的穿刺伤害。如果成功,伤害减半。

当你达到该职业等级的18级时,该穿刺伤害增加到2d6。

\subparagraph{追踪箭 Seeking Arrow}使用预言魔法,你赋予你的箭自寻目标的能力。当你使用该选项时,你不用为该次攻击投攻击骰,取而代之,选择一名在过去的1分钟里你见过的生物。如果需要的话,这只箭会立刻掉头飞向那名生物,且忽视四分之三和半遮蔽。如果目标在箭的射程内且存在容许箭飞行至目标的路程,目标必须进行一次对抗你奥数射击DC的敏捷豁免。如果不满足条件,箭会在飞行到最大距离后消失。豁免失败者将会受到伤害,该伤害等同被这只箭击中的伤害额外附加1d6的穿刺伤害,而且你得知目标的确切位置。如果豁免成功,伤害减半且你无法得知其方位。

当你在该职业等级提升到18级时,该穿刺伤害增为2d6。

\subparagraph{暗影箭 Shadow Arrow}你施加幻术魔法到你的箭上,使其能用暗影蒙蔽敌人的视野。被这发箭击中的生物会额外受到2d6的精神伤害,而且目标必须进行一次对抗你奥数射击DC的感知豁免,豁免失败者无法看见5尺之外的任何事物,直到你的下一个回合开始。

当你在该职业等级提升到18级时,该精神伤害增为4d6。

\subsection{骑兵 Cavalier}
这种泛型擅长乘骑作战。他们通常出身于贵族,成长在宫廷,骑兵往往在家族中领导骑兵队冲锋陷阵或在国家庆典中担任仪仗或守卫。他们学习如何在冲锋时保护他人免受伤害,经常以保护者的身份服从长官或者服务弱者。惩奸除恶或赚取声望,许多骑兵远离了曾经的安逸生活,而踏上了荣耀的冒险之途。

\begin{dndtable}[cX]
\textbf{等级} & \textbf{特性} \\ 
3rd & \emph{\specialcell{附赠熟练项 Bonus Proficiencies,\\ 戎马生涯 Born to the Saddle}} \\
7th & \emph{坚定之印 Unwavering Mark} \\
10th & \emph{坚守战线 Hold the Line} \\
15th & \emph{野蛮冲撞 Ferocious Charger} \\
18th & \emph{警戒守卫 Vigilant Defender} \\
\end{dndtable}

\subsubsection{附赠熟练项 Bonus Proficiencies}第3级时,你选择以下技能中的一个并获得其熟练:动物驯服,历史,洞察,表演和游说。或者,你学会一门你选择的语言。

\subsubsection{戎马生涯 Born to the Saddle}第3级起,你作为骑兵的资质是显露无疑。在进行避免从你坐骑上摔下的豁免中,你具有优势。若你从你的坐骑上摔落且摔落不超过10尺,若你没有失能则可以你能够双脚着陆。

最后,骑上或离开一个生物只需要你5尺的移动力,而非你速度的一半。

\subsubsection{坚定之印 Unwavering Mark}第3级起,你能够恐吓你的敌人,抑制他们的攻击并惩罚他们伤害他人的行为。当你以一次近战武器攻击击中一个生物时,你可以将其标记至你的下一个回合,若你失能,死亡或者有其他人标记该生物,此效应中止。

在你的5尺之内,被你标记的敌人的攻击如果不以你为目标,其攻击骰承受劣势。

另外,如果被你标记的敌人对除你以外的人造成伤害,作为一个附赠动作,在你的下一个回合你可以对该敌人进行一次特殊的近战攻击,你这次攻击骰具有优势,且如果击中,攻击武器对目标额外造成等同于你战士等级一半的伤害。

无论你标记了多少名生物,你能够使用的特殊攻击次数等同于你的力量调整值(至少为1),完成一次长休后,你将恢复所有的使用次数。

\subsubsection{防护战术 Warding Maneuver}第7级时,你学会如何抵御针对你、你的坐骑和你附近生物的直接攻击。如果你或者在你5尺之内对你可见的一名生物被一次攻击击中,如果你手持一把近战武器或者一面盾牌,你能以一个反应投掷一个1d8,并将骰值加入被攻击生物的AC中,如果这次攻击仍然击中,被攻击目标对这次攻击伤害有抗性。

你能够使用该特性的次数等同于你的体质调整值(至少为1),完成一次长休后,你将恢复所有的使用次数。

\subsubsection{坚守战线 Hold the Line}第10级时,你成为了封锁敌人的大师。目标在你的攻击范围里移动了5尺以上的距离时,会承受来自于你的借机攻击,且如果你用借机攻击命中敌人,当前回合结束前敌人的移动速度降为0。

\subsubsection{野蛮冲撞 Ferocious Charger}第15级时,你都能够冲击你的敌人,无论你乘骑与否。如果你在攻击一名生物前至少直线移动10尺且这次攻击命中,目标必须进行一次力量豁免(DC等于8+你的熟练调整值+你的力量调整值),豁免失败者被击倒,每到你的回合你都可以使用一次该特性。

\subsubsection{警戒守卫 Vigilant Defender}18级起,你能用你的无与伦比警戒来造成伤害。战斗中,你获得一个特殊的反应,该反应允许你在除你以外的每个生物的回合中,对其进行一次借机攻击,若你在该回合已经使用过普通的反应动作,则你无法使用该特殊的反应动作。

\subsection{武士 Samurai}
武士是一种以坚决的战斗意志击溃对手的战士。他的信念几乎坚不可摧,挡着他道的敌人只有两种选择:投降,或者死亡。

\begin{dndtable}[cX]
\textbf{等级} & \textbf{特性} \\ 
3rd & \emph{\specialcell{习熟练 Bonus Proficiencies,\\ 战意起 Fighting Spirite}} \\
7th & \emph{雅臣言 Elegant Courtier} \\
10th & \emph{不懈志 Tireless Spirit} \\
15th & \emph{迅捷击 Rapid Strike} \\
18th & \emph{向死生 Strength Before Death} \\
\end{dndtable}

\subsubsection{习熟练 Bonus Proficiencies}第3级时,你获得以下你选择的技能熟练:历史、洞察、表演和游说。或者,你学会一门你选择的语言。

\subsubsection{战意起 Fighting Spirit}第3级起,你在战斗中的气势能保护你并帮助你进行精准的攻击。作为一个附赠动作,你能在你的回合内,令你的武器攻击骰在回合结束前带有优势。若你这么做,你同时获得5点临时生命。当你在该职业等级提升时,额外生命的数值也将提升,第10级时为10点,第15级时增加为15点。
你能使用该特性3次,完成一次长休后,你将恢复此特性的所有使用次数。

\subsubsection{雅臣言 Elegant Courtier}第7级起,你对于细节的严谨和专注让你能胜任各种公共场合。当你进行魅力(游说)检定时,你获得等同于你的感知调整值的额外加值。

你的自控能力也让你获得了感知豁免的熟练项。如果你已经有这项熟练,你选择并获得智力或者魅力的豁免熟练项代替之。

\subsubsection{不懈志 Tireless Spirit}第10级起,如果你投先攻时没有剩余的战意使用次数,你恢复一次使用机会。

\subsubsection{迅捷击 Rapid Strike}第15级起,你学会如何放弃精准进行迅捷的打击。如果你在这回合进行攻击动作且且对目标的攻击带有优势,你可以放弃优势转而对目标进行一次额外的武器攻击以作为该动作的一部分。每回合你都只能使用一次该特性。

\subsubsection{向死生 Strength Before Death} 18级起,你的战意能延缓死亡的到来。如果你受到会将你的生命值降到0且没有立刻杀死你的伤害,你可以使用你的反应来延缓倒地昏迷,而且你立刻获得一个额外的回合并打断当前回合。当你在这个额外回合生命值变为0时,承受伤害仍旧如往常一样造成失败的死亡豁免骰,而且3个失败的死亡豁免仍旧会杀死你。当这个额外回合结束时,如果你还是0生命值,你倒下并昏迷。
你必须完成一次长休才能再次使用此特性。