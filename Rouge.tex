%!TEX root = Mainbook.tex
\clearpage
\chapter{游荡者 Rouge}
\begin{quote}
\emph{人们已经忘却进入一个满是尘埃的墓穴的目的,是为了带出埋藏于中的宝物。战斗是留给愚者的,死人可消费不了他们的财富。——Barnabas Bladecutter}
\end{quote}

在蛮力无法完成任务、魔法不可用或不合适时,游荡者则脱颖而出。凭借潜行、诡计与欺骗这些技能上的天赋,游荡者可以在其他角色无法效仿的地方来去自由,恍若无人之境。

一些转投冒险的游荡者曾是罪犯,他们认为躲避怪物比挑战法律底线更为可取。另一些则是职业杀手,他们用自己的天赋来谋求利益可观的合约。还有些人仅仅是热爱战胜眼前挑战所带来的快感。

在冒险的过程中,游荡者可能会制定看起来较为谨慎的做法——很少有盗贼喜欢战斗——同时也会表现出对战利品的热切渴望。大多数时候,在一个游荡者的脑海中,拿起武器对付一个生物不是为了将它击杀,而是要成为它财宝的新主人。

下面几节从几个方面探讨了成为游荡者意味着什么。你可以用它来增加你角色扮演上的深度。

\section{暗昧之欢 Gulity Please}游荡者们大都是在偷宝探宝间相互竞争;很少有东西能够阻碍他们实现目标,不过很多游荡者都会被一时的冲动蒙蔽住双眼而偏离计划——那是一种必须被满足的需求,即使意味着以身犯险。

一个游荡者的暗昧之欢可能是获得某件实际的物品、想要体验的某种经历,或者干脆就是强迫症。某个游荡者可能不会放过任何银制的战利品,即使那东西挂在某个城堡守卫的脖子上。另一个无法忍受在城市里度过一天而不偷一两个钱包,仅仅是为了能够熟能生巧。

当机会主动送上门时,你的游荡者角色不惜任何代价,即使会为你和你的同伴带来麻烦,也无法抗拒的一种诱惑是什么?

\begin{dndtable}[cX]
\textbf{d6} & \textbf{欢愉} \\
1 & \emph{大宝石} \\
2 & \emph{美丽的脸庞上的微笑} \\
3 & \emph{适合你的一枚新指环} \\
4 & \emph{打击他人自尊的机会} \\
5 & \emph{上好的佳肴与酒水} \\
6 & \emph{在你的藏品中增添一枚奇异的硬币} \\
\end{dndtable}

\section{对手 Adversary} 自然地,遵纪守法的人难免会遇到违背法律的人。而很少有游荡者没有被印在至少一张通缉令上。除此之外,出于职业性质,游荡者会经常与犯罪分子接触,无论是否出于自愿或必须。其中一些人也可能成为你的对手,而且他们可能比城市守卫更难对付。

如果你角色的背景故事中还没有包含这样一个人物,你可以和你的DM一起协商,找出一个对手出现在你人生中的原因。也许你已经被一个出于恶毒目的想要利用你的人监视了一段时间。现在,你已经知道了。诸如此类的事件,可以是你后续冒险的基石。

你的游荡者角色是否有一个对手,也碰巧是个罪犯?如果是,这种关系如何影响你的生活?

\begin{dndtable}[cX]
\textbf{d6} & \textbf{对手} \\
1 & \emph{一名你曾经作为船员,为之效力的海盗船长——你的离去被他视为背叛} \\
2 & \emph{一名因为你的错误情报,导致错误的暗杀目标被刺的间谍大师} \\
3 & \emph{当地盗贼工会的会长,他要你加入组织或滚蛋} \\
4 & \emph{一名用非正当手段获取名作的艺术收藏家} \\
5 & \emph{一个用你来当信使发起违法集会的销赃犯} \\
6 & \emph{一个你曾经在那下过注的地下竞技场的所有者} \\
\end{dndtable}

\section{恩人 Benefactor}很少有游荡者在得到他人的救济之前能把日子过的安稳。这也意味着在此之后,你会欠你的救济者一笔重情。

如果你角色的背景故事中还没有包含这样一个人物,你可以和你的DM一起协商,找出一个恩人出现在你人生中的原因。也许你因为某些事情从你的恩人那里受益,你却不知道这个人是谁。而现在你知道了这个人的身份。是谁在过去帮助了你,无论当时你是否知晓,你又以何作为报答?
\begin{dndtable}[cX]
\textbf{d6} & \textbf{恩人} \\
1 & \emph{一名防止了你被捉住,但为此损失了一批贵重货物的走私犯} \\
2 & \emph{一位多次将你从追踪者的面前藏匿起来,以作为对将来回报投资的乞丐王} \\
3 & \emph{一位帮你免除牢狱之灾,以换取一名强大犯罪首脑的情报的地方法官} \\
4 & \emph{在你年轻的时侯用他们的积蓄帮你摆脱了困境,现在穷困潦倒了的父母} \\
5 & \emph{一条在有机会的时候没有吃掉你,让你答应为它预留一些珍贵的财宝作为回报的龙} \\
6 & \emph{一名曾经帮助你走出困境的德鲁伊。现在你看到的任何动物都可能是当年帮助过你的那个人,也许他/她是来索求回报的} \\
\end{dndtable}

\section{游荡者泛型 Roguish Archetypes}
当你到达第3级时,游荡者将获得其游荡者泛型特征。下列选项给出了在玩家手册提供的泛型以外的额外泛型选项:探查者,策士,斥候和游荡剑客。

\subsubsection{探查者 Inquisitive}作为一名好奇宝宝,你精于揭开秘密,开解谜团。你不光用锐利的双眼去寻找细节,也使用磨练过的高超技艺去研读其他生物的言辞和表情,以发现他们真实的意图。你娴于找出并击败那些藏身于平民之间,也以平民为食的怪物,而你的对于知识的掌握和锐利的眼目也让你更好地揭示并终结潜藏的邪恶。

\begin{dndtable}[cX]
\textbf{等级} & \textbf{特性} \\
3rd & \emph{\specialcell{听声辨谎 Ear for Deceit,\\细致入微 Eye for Detail,\\战斗洞察 Insightful Fighting}} \\
9th & \emph{镇定之眼 Steady Eye} \\
13th & \emph{精准之眼 Unerring Eye}\\
17th & \emph{识破弱点 Eye for Weakness}\\
\end{dndtable}

\subsubsection{听声辨谎 Ear for Deceit}你第3级选择此范型时,你的双耳变得锐利,能够辨识谎言。当你进行一次感知(洞察)检定来察觉一个生物是否在说谎时,你可以将d20投掷结果为7或更低的结果都视为8。

\subsubsection{细致入微 Eye for Detail}第3级起,你可以使用附赠动作来进行一次感知(观察)检定来侦测到一个隐藏的生物或物体,或者是进行一次智力(调查)检定来揭示和破译线索。

\subsubsection{战斗洞察 Insightful Fighting}第3级时,你得到了解到一个敌人的战术并加以反制的能力。以一个附赠动作,你对于一个你能看见的,并非失能的生物进行一次感知(洞察)检定,对抗目标的魅力(欺瞒)检定。若你成功,你可以在你并未对该生物具有优势时对他发动偷袭,但在你的攻击骰有劣势时不行。这个效果持续1分钟或直到你成功的对另一个目标使用了战斗洞察为止。

\subsubsection{镇定之眼 Steady Eye}第9级时,你可以在进行任何感知(观察)检定或智力(调查)检定中得到优势,只要你在同一回合中移动不超过一半的速度。

\subsubsection{精准之眼 Unerring Eye}第13级时,你得到侦测魔法欺瞒的能力。作为一个动作,你感知到你周围30尺内的幻术存在,不在天生形态下的变形生物,和其他意在欺瞒感官的魔法,前提是你没有目盲或耳聋。你能了解到了有一个效果正在试图欺骗你,不过你并没有得到对其后所掩盖的真相和对他真实本质的特殊洞察力。

你可以使用这个功能的次数等于你的感知调整值(至少为1),完成一次长休后,你将恢复所有的使用次数。

\subsubsection{识破弱点 Eye for Weakness}
第17级时,你学会通过认真的研究一个生物的战术和移动来找到他的弱点的能力。当你的战斗洞察对一个生物使用时,你对该生物的偷袭伤害提高3d6。

你可以使用这个功能的次数等于你的感知调整值(至少为1),完成一次长休后,你将恢复所有的使用次数。

\subsection{策士 Mastermind}
策士专注于对人们施加影响和知晓他们的秘密。许多间谍,朝臣和阴谋家会追随这个范型,在密谋中渡过他们全部的生命。言语像匕首和毒药一般作为你的武器,通过它们你得到你最喜欢的宝藏——人们的要害所在以及他们言不由衷的支持。

\begin{dndtable}[cX]
\textbf{等级} & \textbf{特性} \\
3rd & \emph{\specialcell{阴谋大师 Master of Intrigue,\\战术大师 Master of Tactics}} \\
9th & \emph{洞察力 Insightful Manipulator} \\
13th & \emph{误导 Misdirection}\\
17th & \emph{心灵欺骗 Soul of Deceit}\\
\end{dndtable}

\subsubsection{阴谋大师 Master of Intrigue}第3级起,你可以得到易容工具、文书伪造工具和一种自选赌博工具的熟练。你同时可以学会两种自选语言。

此外,如果你听到一个生物讲话长于1分钟,你可以确切无误地模仿他的语气和口音。这允许你模仿来自某个特殊地区的特别口音,只要你知晓这种语言。

\subsubsection{战术大师 Master of Tactics}第3起起,你可以把协助动作作为附赠动作使用。此外,当你使用协助动作帮助一个盟友攻击一个生物时,如果目标可以看到你或者听到你的话,攻击的目标可以在30尺以内,而不是你的5尺之内。

\subsubsection{洞察力 Insightful Manipulator}第9级起,如果你至少花1分钟观察或与战斗之外的其他生物进行交互,你可以了解到跟你自己的能力相比较的某些信息。DM告诉你,在你选择的下列两个特征中,是否你的相等、优越或低等:
\begin{itemize}
\item 智力值
\item 感知值
\item 魅力值
\item 职业等级(如果有的话)
\end{itemize}

如果你的DM允许,你也有可能知道这个生物的一段经历或者它的一个性格特征,如果它有的话。

\subsubsection{误导 Misdirection}第3级起,你的急才有时候能让另一个生物代替你承受攻击。每当你成为一次攻击的目标,并且在你的5尺距离内有生物为你对抗这次攻击提供了掩蔽,你可以使用你的反应将这次攻击的目标转移给这个生物。

\subsubsection{心灵欺骗 Soul of Deceit}第17级起,除非你允许,否则你的思想不能被心灵感应或者其他能力阅读。如果你使用魅力(欺瞒)检定并在与思想阅读者的感知(洞悉)检定的对抗之中胜出,你还可以伪造思想使对方信以为真。
此外,不管你说什么,你总是可以选择让魔法判定你在说真话并显现出相应的迹象。你也不会因为魔法效应的影响而被强迫说出真话。

\subsection{斥候 Scout}你擅长在远离城市和街道的地方潜行和生存,这让你在探险中能离开你的同伴带头侦查。这一泛型的斥候许多都以荒野为家,与野蛮人和游侠同行,也有许多斥候充当着战争地带的眼睛和耳朵。伏击兵,间谍,赏金猎人——这些只是斥候在世界范围内所扮演的少数角色。

\begin{dndtable}[cX]
\textbf{等级} & \textbf{特性} \\
3rd & \emph{\specialcell{散兵战术 Skirmisher,\\生存专家 Survivalist}} \\
9th & \emph{优异灵活 Superior Mobility} \\
13th & \emph{突袭大师 Ambush Master}\\
17th & \emph{突袭打击 Sudden Strike}\\
\end{dndtable}

\subsubsection{散兵战术 Skirmisher}第3级起,你在战斗中变得难以被压制。当一个敌人在你周围5尺内结束其回合时,你可以以一个反应移动至多你速度的一半。此移动不会导致借机攻击。

\subsubsection{生存专家 Survivalist}你第3级时选择此范型时,你得到对于自然和生存技能的熟练。当使用上述熟练进行的任何属性检定中,你的熟练加值加倍。

\subsubsection{优异灵活 Superior Mobility}第9级时,你的行走速度增加10尺。若你具有攀爬或游泳速度,这个能力也会增加在这些速度上。

\subsubsection{突袭大师 Ambush Master}第13级起,你精通于引领突袭,在战斗中首先行动。

你在先攻检定时具有优势。此外,在第一轮战斗中你所命中的第一个生物对你和其他人而言会变得更容易打击;直到你下一回合开始之前,所有对它进行的攻击检定具有优势。

\subsubsection{突袭打击 Sudden Strike}
第17级起,你可以以致命的速度进行打击。若你在你的回合进行攻击动作,你可以以一个附赠动作进行一次额外的攻击。此攻击可以得到你偷袭的收益,即使你已经在此回合中使用了它,但是你不能在一回合中对同一目标偷袭一次以上。

\subsection{游荡剑客 Swashbuckler}你全身心投入剑术的训练,一种兼顾了速度、优雅和魅力的剑术。对比其他被重甲包裹成粗蛮罐头的战士,你战斗的方式看起来更象在表演。扫荡者、决斗家和海盗是这类的典型。

游荡剑客擅长一对一的战斗,也精通于无所顾虑地使用双武器攻击对手,在那以后飞快地从他身边逃脱。

\begin{dndtable}[cX]
\textbf{等级} & \textbf{特性} \\
3rd & \emph{\specialcell{梦幻舞步 Fancy Footwork,\\胆大无畏 Rakish Audacity}} \\
9th & \emph{潇洒气质 Panache} \\
13th & \emph{优雅战技 Elegant Maneuver}\\
17th & \emph{决斗大师 Master Duelist}\\
\end{dndtable}

\subsubsection{梦幻舞步 Fancy Footwork}第3级起,你冲刺,攻击,摆脱这些战斗中的动作就像一串模糊的身影。在你的回合,如果你对一个生物进行一次近身攻击,那个生物在你回合剩下的时间内无法对你进行借机攻击。

\subsection{胆大无畏 Rakish Audacity}第3级起,你万无一失的自信驱使着你去战斗。你将你的魅力调整值加入你的先攻掷骰值。

你也可以通过一种额外的方式来使用偷袭攻击:只要你在目标生物周围5尺内,并且没有其他生物在你5尺内,你的攻击检定也没有劣势,你就可以无需攻击检定的优势对目标生物进行偷袭。其他所有关于偷袭的规则依然适用于你。

\subsubsection{潇洒气质 Panache}第9级起,你的魅力变得更加迷人。你可以用一个动作进行一次魅力(游说)检定来对抗一个生物的感知(洞察)检定。目标生物必须能听到你的声音,并且你们两个必须会同一种语言。

如果那个生物是敌对的并且你的检定成功,它对除你之外的其他目标攻击取劣势,并且不能对除你之外的目标进行借机攻击。这个效果持续1分钟,或你的队友攻击/用法术影响了这个目标,或你移动到远离目标60尺外为止。

如果那个生物不是敌对的并且你的检定成功,它被你魅惑1分钟。当它被魅惑时,它认为你是一个友好的熟人。这个效果会在你或者你的队友做出任何对他有害的行为之后立即结束。

\subsubsection{优雅战技 Elegant Maneuver}第13级起,你可以轻松熟练的完成困难的战技。你可以在你的回合用一个附赠动作来获得在同一回合中下一个敏捷(体操)或力量(运动)检定的优势。

\subsubsection{决斗大师 Master Duelist}第17级起,你对剑术的精通使你可以扭转战斗的胜负。如果你的一次攻击失手,你可以选择以优势再投一次攻击骰。你必须完成一次短休或长休才能再次使用此特性。