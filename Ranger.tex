%!TEX root = Mainbook.tex
\clearpage
\chapter{游侠 Ranger}
\begin{quote}
\emph{为了守护文明,我一生大多数的时间都游走于它的边界来守护它。不要以为我不向你的国王效忠,我的贡献就比所有保护他的骑士加起来少。
——Soveliss}
\end{quote}

游侠是一群无忧无虑的漫游者和搜寻者,他们出没于文明世界的边缘,抵御外域居民。这是一份无偿的工作,自从他们的事业鲜有人理解且不计回报。而且游侠们恪尽职守,坚信他们的工作会让世界变得安全。

与文明世界之间的关系会体现每个游侠的个性和经历。有些游侠视自己为不受王权约束的,文明边境的法律与正义的使者。其他则是完全规避文明社会的生存主义者。当他们居住和游历于世界上危险的荒野之地时,他们会为了保护自己而猎杀怪物。如果他们的所作所为对他们躲避的王国和文明地区有益,他们也照做不误。

如果你想创造或者游玩一个游侠角色,以下选项能提供一些办法来丰富角色和增加你的表演经验。

\section{世界观 View of the World}一个游侠的世界观开始(有时候结束)于角色对文明种族和他们的领地的看法。有些游侠对文明抱有一种根深蒂固的偏见,当其他人同情那些他曾发誓要保护的人时,即使是在战场上,它也有可能体现出两个游侠之间的不同。事实上,为了那些看到过自己的卓越技艺并能从他们的力量中受益的人,游侠完全可以自作主张。然而即便如此,没有两个游侠会以同一种方式来体现他们的特性。

如果你还未想好你的角色世界观的具体内容,可以考虑通过一个简短观点(如下表所例)的概述来提取要点。你的情感如何影响你的行为?

\begin{dndtable}[cX]
\textbf{d6} & \textbf{观点} \\
1 & \emph{城镇和城市是那些无法自我生存的人最好的去处。} \\
2 & \emph{文明进步的最好办法就是阻止混乱,但它的范围必须被监视。} \\
3 & \emph{城镇和城市是必要的,可一旦荒野的超自然威胁解除了,我们将不再需要他们。} \\
4 & \emph{城墙是为那些当别人还在为和平奋斗自己却躲在后面的懦夫准备的。} \\
5 & \emph{刚开始拜访一处城镇还不错,但没过几天我就会因感受到不可抗拒的呼唤而回归山野。} \\
6 & \emph{城市会因将族群从野外的艰难挑战中隔离开来而产生弱点。} \\
\end{dndtable}

\section{家乡 Homeland}所有的游侠,无论他们如何开始从事自己的专业,对自然世界和多样的地貌都会有一种强烈的联系。对于一些游侠而言,荒野是他们成长的地方,也是他们出生于此或者早年移居于此的结果。对于其他游侠而言,文明社会是原本的家,而荒野是他们的第二个家。

考虑你角色的背景故事并决定最想家的地方,以及你是否出生在那里。对于你个人来说那个地方叫什么名字?它是否影响你学习什么样的法术?你在那里的经历是否决定了你的宿敌?

\begin{dndtable}[cX]
\textbf{d6} & \textbf{家乡} \\
1 & \emph{你曾巡逻过一片被些许穿插于其中的暗影堕界所污染和毁坏的远古森林。} \\
2 & \emph{作为游牧族群的一份子,你的荒野生存技艺炉火纯青。} \\
3 & \emph{你的早期在幽暗地区的生活让你对殴打别人有了充足的准备。} \\
4 & \emph{你居住在一片澡泽的边缘,那里被一种水生的存在所威胁。} \\
5 & \emph{因为你成长于山顶,寻找最好的路来穿山越岭是你的本能。} \\
6 & \emph{你漫游过北境,知道如何保护自己并在冰天雪地里生存。} \\
\end{dndtable}

\section{宿敌 Sworn Enemy}每个游侠都有宿敌。宿敌的决定因素可源自于角色早期的特殊事件,或者完全取决于选择。

什么鞭策你选择了一个特殊的敌人?做这个选择是因为传统还是好奇心,或者仇恨?

\begin{dndtable}[cX]
\textbf{d6} & \textbf{宿敌} \\
1 & \emph{为了你的敌人曾犯下的大罪,你在自然的支持下需求复仇。} \\
2 & \emph{你的祖先或者前辈曾和某种生物作战过,现在轮到你了。} \\
3 & \emph{你对你的敌人没有丝毫恨意。你只是像一个猎人追踪猎物一样靠近别人。} \\
4 & \emph{你沉迷于搜寻你对敌人,你还搜集与它相关的传说和历史书籍。} \\
5 & \emph{你从你的敌人身上搜集标志以来纪念你的每一次屠杀。} \\
6 & \emph{你尊重你所选择的敌人,而且你视你的战斗为各项技巧的测试。} \\
\end{dndtable}

\section{游侠范型 Ranger Archetype}第3级时,游侠获得选择一种原初道途。除了在中玩家手册所提到的,本书还提供了:幽域追踪者范型、边界行者范型和怪物杀手范型供玩家使用。

\subsection{幽域追踪者 Gloom Stalker}幽域追踪者居身于致暗之地:地表深处,暗巷之间,太古丛林,和其他任何光无法照亮的地方。很多人进入到这种地区都会有种强烈的不安,但是幽域追踪者却能在黑暗中大胆冒险,在敌人能够触及到更开阔的地区之前搜寻并袭击他们。这种游侠经常能在幽暗地区中被找到,但是他们会前往任何邪恶势力蛰伏的地区。

\begin{dndtable}[cX]
\textbf{等级} & \textbf{特性} \\
3rd & \emph{\specialcell{幽域追踪者魔法 Gloom Stalker Magic,\\恐惧奇兵 Dread Ambusher,\\阴影视野 Umbral Sight}} \\
7th & \emph{钢铁意志 Iron Mind} \\
11th & \emph{追猎如风 Stalker`s Flurry} \\
15th & \emph{暗影闪避 Shadow dodge} \\
\end{dndtable}


\subsubsection{幽域追踪者魔法 Gloom Stalker Magic}第3级起,当你到达指定游侠等级时,你学会一个额外法术,如幽域追踪者法术列表所示。这些法术对你视为游侠法术,它不计入你的已知法术总数中。

\begin{dndtable}[cX]
\textbf{等级} & \textbf{法术} \\
3th & \emph{易容术 disguise self} \\
5th & \emph{魔绳术 rope trick} \\
9th & \emph{恐惧术 fear} \\
13th & \emph{高等隐身术 greater invisibility} \\
17th & \emph{伪装术 seeming} \\
\end{dndtable}

\subsubsection{恐惧奇兵 Dread Ambusher}第3级起,你精通于伏击的艺术。你获得一个等同于你的感知调整值的加值在你的先攻投上。

在每一场战斗的你的第一个回合的开始,你的移动速度增加10尺,直到该回合结束。如果你在该回合进行攻击动作,你可以做一次额外的武器攻击作为该动作的一部分。如果该攻击命中,目标受到额外1d8的武器伤害。

\subsubsection{阴影视野 Umbral Sight}第3级起,你获得60尺黑暗视觉。如果你已经从种族特性中获得了黑暗视觉,它的范围增加30尺。

你还擅长回避依赖黑暗视觉的角色。在黑暗中,你对于任何靠黑暗视觉观察你的角色不可见。

\subsubsection{钢铁意志 Iron Mind}第7级时,你已经磨炼出用来抵御来自于你的猎物的精神影响能量的能力。你获得感知豁免熟练项。如果你已经有这项熟练,你可以选择并获得智力或者魅力豁免熟项来代替。

\subsubsection{追猎如风 Stalker`s Flurry}第11级时,你学会以出人意料的速度来发动攻击以至于你能够将失手转化为另一次攻击。你的每个回合一次,当你的一次武器攻击未命中时,你可以做出另一次武器攻击作为同一动作的一部分。

\subsubsection{暗影闪避 Shadow dodge}第15级起,伴随着一缕围绕在你身边的神奇暗影,你能以某种让人始料未及的方式躲闪。无论何时一名角色对你做出一次攻击投掷且没有优势,你都能用反应对其施加劣势。你必须在知道攻击投掷的结果之前使用该特性。


\subsection{边界行者 Horizon Walker}边界行者会保护世界不受其他位面的威胁或者通过其他位面的魔法来找到并破坏其他有威胁的位面,他们会找出传送门并看守他们,会为了追捕他们的敌人而穿梭于表里位面。这些游侠也是多元宇宙中任何势力的盟友,尤其是那些守护位面的生灵和秩序的善良龙类,精怪和元素。

\begin{dndtable}[cX]
\textbf{等级} & \textbf{特性} \\
3rd & \emph{\specialcell{边界行者魔法 Horizon Walker Magic,\\侦测传送门 Detect Portal,\\位面战士 Planar Warrior}} \\
7th & \emph{灵界漫步 Ethereal Step} \\
11th & \emph{闪现打击 Distant Strike} \\
15th & \emph{虚体防御 Spectral Defense} \\
\end{dndtable}

\subsubsection{边界行者魔法 Horizon Walker Magic}第3级起,当你到达指定游侠等级时,你学会一个额外法术,如边界行者法术列表所示。这些法术对你视为游侠法术,它不计入你的已知法术总数中。

\begin{dndtable}[cX]
\textbf{等级} & \textbf{法术} \\
3th & \emph{防护善恶 protection from evil and good} \\
5th & \emph{迷踪步 misty step} \\
9th & \emph{加速术 haste} \\
13th & \emph{放逐术 banishment} \\
17th & \emph{传送法阵 teleportation circle} \\
\end{dndtable}

\subsubsection{侦测传送门 Detect Portal}第3级时,你获得侦测传送门的魔法灵光的能力。以一个动作,在离你1英里的范围内,你侦测到离你最近的传送门的距离和方向。

你必须完成一次短休或长休才能再次使用此特性。

\subsubsection{位面战士 Planar Warrior}第3级起,你学会通过吸取多元宇宙的能量来强化你的攻击。作为一个附赠动作,你选择一名在你30尺范围内对你可见的角色,这回合你对其的下一次武器攻击造成的伤害变成力场伤害,且该攻击对目标额外造成1d8力场伤害,当你达到该职业等级的18级时,额外伤害增为2d8.

\subsubsection{灵界漫步 Ethereal Step}第7级时,你学会如何进入灵界位面。作为一个附赠动作,你能够利用这个特性不消耗法术位施放\emph{同游灵界 etherealness},但是这个法术会在当前回合结束时结束。

你必须完成一次短休或长休才能再次使用此特性。

\subsubsection{闪现打击 Distant Strike}第11级时,你能够在眨眼之间穿梭于两个位置。当你进行攻击动作时,你能够在每一次攻击前传送10尺到一片对你可见的空地上。

如果你用一个动作攻击了至少两名不同的角色,你可以额外攻击一次第三名角色。

\subsubsection{虚体防御 Spectral Defense}第15级时,你的空间传送能力能让你在战斗中流转于空间边缘以减少伤害。当你受到一次攻击伤害时,你可以用你的反应来给予你对那次攻击全部伤害的抗性在这一回合。


\subsection{怪物杀手 Monster Slayer}你专注于猎杀那些黑夜中的怪物和邪恶魔法的使用者。怪物杀手会找出吸血鬼,龙,邪恶精灵,魔裔以及其他魔法威胁。为了训练出高超技巧来击败这些怪物,怪物杀手精通于找出并击败那些强大且神秘的敌人。

\begin{dndtable}[cX]
\textbf{等级} & \textbf{特性} \\
3rd & \emph{\specialcell{怪物杀手魔法 Monster Slayer Magic,\\猎人感知 Hunter`s Sense,\\狩猎开始 Slayer's Prey}} \\
7th & \emph{卓越防守 Supernatural Defense} \\
11th & \emph{魔法克星 Magic-user's Nemesis} \\
15th & \emph{杀手反制 Slayer`s Counter} \\
\end{dndtable}

\subsubsection{怪物杀手魔法 Monster Slayer Magic}第3级起,当你到达指定游侠等级时,你学会一个额外法术,如怪物杀手法术列表所示。这些法术对你视为游侠法术,它不计入你的已知法术总数中。

\begin{dndtable}[cX]
\textbf{等级} & \textbf{法术} \\
3th & \emph{防护善恶 protection from evil and good} \\
5th & \emph{诚实之域 zone of truth} \\
9th & \emph{防护法阵 magic circle} \\
13th & \emph{放逐术 banishment} \\
17th & \emph{怪物定身术 hold monster} \\
\end{dndtable}

\subsubsection{猎人感知 Hunter`s Sense}第3级起,你获得了观察生物并魔法性地分析出最佳狩猎方式的能力。作为一个动作,你选择一名60尺范围内对你可见的角色。你立刻知晓该角色的伤害免疫,抗性和易伤。如果该角色被预言系魔法覆盖,它对你如同没有任何伤害免疫,抗性和易伤。

你能使用该特性的次数等同于你的感知调整值(至少为1)。完成一次长休后,你将恢复所有的使用次数。

\subsubsection{狩猎开始 Slayer's Prey}第3级起,你能够专注于你的一名敌人,来增加你对它的伤害。以一个附赠动作,你选择一名60尺范围内对你可见角色作为该特性的目标。每回合你以一次武器攻击第一次击中该目标,它额外受到1d6的武器伤害。

该收益持续到你完成一次短休或者长休。如果你选择一名不同的目标,它会提前结束。

\subsubsection{卓越防守 Supernatural Defense}第7级时,对于你的猎物对你的身体和精神的攻击,你拥有额外抗性。无论何时你的狩猎开始特性目标迫使你作出一个豁免或者迫使你做一次技能检定来脱离目标的擒抱,你的检定值额外增加1d6。

\subsubsection{魔法克星 Magic-user's Nemesis}第11级时,你获得打断其他人的魔法的能力。60尺内,当你看到一名角色使用法术或者传送,你可以用你的反应魔法性地来尝试打断它。角色必须成功通过一个对抗你法术DC的感知豁免,否则它的法术和传送失败并浪费。

你必须完成一次短休或长休才能再次使用此特性。

\subsubsection{杀手反制 Slayer`s Counter}第15级时,当你的猎物企图干扰你时你能够反击。如果你的狩猎开始特性目标使你做出一个豁免投,你能够使用你的反应来对其做出一次武器攻击。你在进行豁免检定前立刻做出该攻击。如果你击中,你的豁免自动成功,否则只造成普通攻击效果。