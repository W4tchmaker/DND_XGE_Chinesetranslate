%!TEX root = Mainbook.tex
\clearpage
\chapter{圣武士 Paladin}
\begin{quote}
\emph{衡量一个圣武士的价值,绝不在于计算他掠夺地城或打败敌人的数量。圣武士的真正价值,在于他拯救的生命,以及他对于正义仁慈之事业的决心。——Isteval}
\end{quote}

一个圣武士是其誓言的鲜活象征,且这位神圣的斗士有着能力与决心见证自己誓言的终结。一些圣武士专注于保护无辜者并以此在世间贯彻正义,而另一些圣武士则通过征服并统治叛逆者来达到这个目标。

虽然世上没有一个圣武士是典型的,但他们中确实有一部分是那种在狭隘定义上的好人,拒绝容忍那些甚至与他们只有毫厘之差的观点。不过,作为冒险者的那些圣武士,则很少保持这种僵硬的态度,哪怕只是为了避免与同伴疏远。

你可以使用下面的一些建议来充实你的圣武士角色。请记住,大多数圣武士并不是机器人,他们的内心中也有怀疑与偏见、纠结与矛盾,就像其他人一样。有些圣武士有自己的想法,而这有时可能会与他们的誓言不一致。

\section{个人目标 Personal Goal}圣武士宣誓的规则为角色提供了一个其所遵循与推进的终极目标,或称整体目标。此外,一些圣武士其实是受其个人的目标驱动的,这项个人目标可以补充甚至超越他们的誓言。遵守不同誓言的圣武士也可能有着相同的个人目标,只是在坚持誓言的过程中,他们以不同的方式将这项个人目标贯彻到行动中。如果你的圣武士角色有一项个人的目标,它也许诞生自一些生活事件,从而并不与誓言直接挂钩。
\begin{dndtable}[cX]
\textbf{d6} & \textbf{个人目标} \\
1 & \emph{和平。你为之奋斗,而后代不必。}\\
2 & \emph{复仇。你的誓言是你纠正一个过往错误的工具。}\\
3 & \emph{职责。你不会辜负你的诺言,至少你拼死确保如此。}\\
4 & \emph{领导。你将赢得一场为人传诵的大战,以此成为激励他人的榜样。}\\
5 & \emph{信念。你知道你的道路是正义的,否则众神不会让你踏入其中。}\\
6 & \emph{荣耀。你将引领世界进入一个新时代,它必被冠以你的名字。}\\
\end{dndtable}


\section{纹饰 Symbol}圣武士注意到了纹饰的影响,他们之中许多人采用甚至亲自设计出一个具有独特形象的艺术纹饰。你的纹饰代表了你的誓言,并向周围的人彰显敌我。你的纹饰也许是光明正大地摆在旗帜、横幅或衣服上给所有人看的,但它也可以不那么明显,就像是隐藏在你身上的私密饰品或标记。

\begin{dndtable}[cX]
\textbf{d6} & \textbf{纹饰} \\
1 & \emph{一条龙。象征你平和的高贵和激烈的奋斗。}\\
2 & \emph{握紧的拳头。因为你随时准备为信仰而战。}\\
3 & \emph{张开的手掌。表明你比起争斗更偏好外交。}\\
4 & \emph{红色的心。向世人展现你对正义的承诺。}\\
5 & \emph{黑色的心。表示怜悯之类的情绪不会影响你对誓言的奉献。}\\
6 & \emph{不眨的眼珠。你对所有可能危害到你的事业的潜在威胁永远保持警惕。}\\
\end{dndtable}

\section{克星 Nemesis}圣武士们严守的誓言要求他们以积极的态度传播他们的信仰,而这种行为自然会与反对这种信仰的生物或实体发生冲突。
在这些对手之中,一名克星——常常是一名圣武士最顽固或最可怕的敌人——的存在或影响是圣武士生命中的常态。你的圣武士角色很可能从起步时就有了这样一位敌人。
也许在你成为圣武士的时候,你就吸引了那些会与你发生冲突的人的注意,而你成为了他们的目标。如果你有一个克星,那他是谁?或者说,是什么?你认为谁是你的敌人,谁是你实现目标的最大威胁?

\begin{dndtable}[cX]
\textbf{d6} & \textbf{克星} \\
1 & \emph{一名强大的兽人酋长,威胁要超越并摧毁你的一切。}\\
2 & \emph{一名恶魔或天族,外层位面力量的代理人,你被指名受其腐败或救赎,视情况而定。}\\
3 & \emph{一条龙,它的仆人缠住了你,阻碍你的步伐。}\\
4 & \emph{大祭司将你看作误入歧途的愚昧之徒,想要你放弃你的信仰。}\\
5 & \emph{一名曾与你一同训练的圣武士,如今背弃了誓言,与你对立。}\\
6 & \emph{被某人打败的吸血鬼,发誓要向所有圣武士复仇。}\\
\end{dndtable}

\section{惑念 Temptation}
尽管圣武士为誓言献身,但他们终究是凡人,有所缺陷。大多数圣武士依然会表现出某种并不符合他们最高理想的态度或行为。你的角色容易向怎样的诱惑屈服?抑或是发现有什么是难以抵挡的?

\begin{dndtable}[cX]
\textbf{d6} & \textbf{惑念} \\
1 & \emph{暴怒。当你被惹火,你就难以冷静思考,甚至可能会做一些日后将会为之懊恼的事情。}\\
2 & \emph{骄傲。你的行径值得注意,没有人比你更注意它们。}\\
3 & \emph{色欲。你无法抗拒一张诱人的脸和一个愉快的笑容。}\\
4 & \emph{嫉妒。你注意到一些著名人士。当你的行为不足以与他们比肩,你就感到不爽。}\\
5 & \emph{绝望。你意识到你要打败的所敌人拥有的强大力量,有时你感觉无法看到哪怕一丝胜利的希望。}\\
6 & \emph{贪婪。无论是多少荣耀与财富,都无法满足你的贪欲。}\\
\end{dndtable}
\section{神圣誓言 Sacred Oaths}在3级时,圣武士将获得神圣誓言特性。下列选项给出了在玩家手册提供的誓言以外的额外誓言选项:征服誓言和救赎誓言。

\subsection{征服之誓 Oath of Conquest}征服的誓言吸引着在战斗中寻求荣耀与征服的圣武士。对他们来说,简单地建立秩序是不够的,混乱的势力必须被粉碎。有些征服圣武士会被称为暴君骑士或战争贩子,他们聚集成严酷的军队或组织,来为秩序与战争的神明或哲学服务。

有些这样的圣武士走得太远,甚至会与九狱合流,他们的眼中法治凌驾于仁慈。大恶魔巴尔,Avernus的战争领主,统领着许多这样的骑士——他们被称为地狱骑士——视为他最热诚的追随者。地狱骑士用倒下的敌人身上的战利品来装饰他们的盔甲,警告着任何敢于反抗他们乃至他们领主的法令的人的可怕下场。这些骑士往往会受到其他征服圣武士的抵制——他们认为地狱骑士在黑暗中走得太远了。

\subsection{征服之信条 Tenets of Conquest}选择征服之誓的圣武士会将征服信条烙印在他们的上臂上。
\subparagraph{掐灭希望 Douse the Flame of Hope}仅仅在战斗中击败敌人是不够的。你的胜利必须显著到彻底击碎敌人的战斗意志。刀剑只能结束生命。恐惧却能终结国家。
\subparagraph{治以铁腕 Rule with an Iron Fist}一旦你完成了征服,不要容忍任何异议。你说出的话就是法律。遵循的人会被偏爱。不服从的人则会受到惩罚,以儆效尤。
\subparagraph{力量至高 Strength Above All}你会一直统治下去,直到更强的势力兴起。到那时,你必须变得更强并面对挑战,否则就迎来自己的毁灭。

\begin{dndtable}[cX]
\textbf{等级} & \textbf{特质} \\
3rd & \emph{\specialcell{圣誓法术 Oath Spells,\\引导神力 Channel Divinity}}\\
7th & \emph{征服灵光 Aura of Conquest(10尺)}\\
15th & \emph{轻蔑呵斥 Scornful Rebuke}\\
18th & \emph{征服光环 Aura of Conquest(30尺)}\\
20th & \emph{不移胜者 Invincible Conqueror}\\
\end{dndtable}
\subsubsection{圣誓法术 Oath Spells}你在下述列出的圣武士等级得到誓约法术。誓约法术如何运作,参见神圣誓约职业特性。

\begin{dndtable}[cX]
\textbf{等级} & \textbf{法术} \\
3rd & \emph{\specialcell{艾嘉西斯之铠 armor of Agathys,\\ 命令术 command}}\\
5th & \emph{\specialcell{人类定身术 hold person,\\ 灵体武器 spiritual weapon}}\\
9th & \emph{\specialcell{降咒 bestow curse,\\ 恐惧术 fear}}\\
13th & \emph{\specialcell{支配野兽 dominate beast,,\\ 石肤术 stoneskin}}\\
17th & \emph{\specialcell{死云术 cloudkill,\\ 支配人类 dominate person}}\\
\end{dndtable}

\subsubsection{引导神力 Channel Divinity}第3级选择本誓言时,你获得以下两种引导神力选项。引导神力如何运作,参见神圣誓约职业特性。
\subparagraph{征服威压 Conquering Presence}你可以使用你的引导神力来发出一阵可怕的威压。以一个动作,你强迫每个处于你周围30尺内、你能看见的、由你选择的生物各进行一次感知豁免。豁免失败者将变得恐慌于你,持续1分钟。变得恐慌的生物可以在其每个回合的结束时重复此豁免检定,成功则结束其身上的此效果。
\subparagraph{神助打击 Guided Strike}你可以使用你的引导神力来以超自然的精确性进行打击。当你进行一次攻击检定时,你可以使用你的引导神力以在此检定中得到+10加值。你在你看到掷骰后进行决定,但需要在DM宣布此攻击命中与否前声明。

\subsubsection{征服灵光 Aura of Conquest}第7级起,只要你并未失能,你便弥漫出一阵威吓的灵光。光环从你开始向各个方向延伸10尺,它会被全掩蔽所阻挡。

若一个生物恐慌于你,当其处于此光环内,其速度降为0,且若其在光环内开始其回合,则受到等于你圣武士等级一半的心灵伤害。

18级时此光环的距离增加到30尺。

\subsubsection{轻蔑呵斥 Scornful Rebuke}从15级起,那些胆敢击中你的人,将在心灵上因为无礼而受到惩罚。当一个生物以一次攻击击中你时,若你并未失能,该生物受到等于你魅力调整值(最小为1)的心灵伤害。

\subsubsection{不移胜者 Invincible Conqueror}在20级时,你得到驾驭特异武艺的能力。以一个动作,你可以魔法性的变成征服的化身,获得下述的增益,持续1分钟:
\begin{itemize}
\item 你对于所有伤害得到抗力。
\item 当你在你的回合进行攻击动作时,你可以进行一次额外的攻击,作为该动作的一部分。
\item 你的近战武器攻击在掷出19或20时即可造成重击。
\end{itemize}

你必须完成一次长休才能再次使用此特性。


\subsection{救赎之誓 Oath Of Redemption}救赎的誓言为圣武士提供了一条艰难的限制:不到最后关头就不能动用暴力。献身于救赎之誓的圣武士相信,任何人都可以得到救赎,仁慈与正义的路应当是所有人都可以走的。这些圣武士直面邪恶的生物,希望使他们弃暗投明,除非是为了在紧要关头救人,否则他们绝不会杀死自己的敌人。循此道者,便被称为救赎者。

救赎圣武士虽然可以说是理想主义者,但他们不是傻瓜。他们也明白不死生物、恶魔、魔鬼和其他一些超自然的威胁从本质上就是邪恶的。对于这些坚定不移的邪恶,遵循救赎之誓的圣武士随时准备用武器和法术来击垮他们。然而尽管如此,救赎圣武士也在心底祈祷着,总有一天,即使是如此的邪恶也能够自我救赎。

\subsubsection{救赎之信条 Tentes Of Redemption}救赎之誓的信条要求圣武士高标准地保持和平与正义。
\subparagraph{和平 Peace}暴力是最后的手段,交流和理解才是持久和平的正解。
\subparagraph{无辜 Innocence}所有人生来都是无辜的,他们走向邪恶往往是环境或某些黑暗力量的影响。通过树立正确的榜样,治愈这个残缺世界的创伤,你可以令任何人走上正确的道路。
\subparagraph{耐心 Patience}改变需要时间。犯下罪行的人必须被诚实与正义时刻提醒。一旦你播下正义的种子,你就必须日复一日地为其努力,直到最终开花结果。
\subparagraph{睿智 Wisdom}你的心灵必须时刻保持清醒,而你最终也许会被迫承认失败。既然所有人都可以得到救赎,那么自然也会有人选择步入邪恶之巅。你别无选择,为了大义只能将其剿灭。任何这样的行动都应当经过仔细权衡并考虑清楚其后果,但你一旦做出决定,那么就做到最后。

\subsubsection{救赎之誓特性}
\begin{dndtable}[cX]
\textbf{等级} & \textbf{特性} \\
3rd & \emph{\specialcell{圣誓法术 Oath Spells,\\引导神力 Channel Divinity}}\\
7th & \emph{护卫灵光 Aura of the Guardian(10尺)}\\
15th & \emph{守护之魂 Protective Spirit}\\
18th & \emph{护卫光环 Aura of the Guardian(30尺)}\\
20th & \emph{救赎使节 Emissary of Redemption}\\
\end{dndtable}

\subsubsection{圣誓法术 Oath Spells}你在下述列出的圣武士等级得到誓约法术。誓约法术如何运作,参见神圣誓约职业特性。

\begin{dndtable}[cX]
\textbf{等级} & \textbf{法术} \\
3rd & \emph{庇护术 sanctuary,睡眠术 sleep}\\
5th & \emph{\specialcell{安定心神 calm emotions,\\ 人类定身术  hold person}}\\
9th & \emph{\specialcell{法术反制 counterspell,\\ 催眠图纹 hypnotic pattern}}\\
13th & \emph{\specialcell{欧提路克弹力法球 Otiluke's resilient sphere,\\ 石肤术 stoneskin}}\\
17th & \emph{\specialcell{怪物定身术 hold monster,\\ 力墙术 wall of force}}\\
\end{dndtable}

\subsubsection{引导神力 Channel Divinity}第3级选择本誓言时,你获得以下两种引导神力选项。引导神力如何运作,参见神圣誓约职业特性。
\subparagraph{和平使节 Emissary of Peace}你可以使用你的引导神力来以神圣力量增强自己的存在。以一个附赠动作,你为自己在接下来的10分钟内进行的下次魅力(说服)检定提供+5加值。
\subparagraph{斥喝暴力 Rebuke the Violent}你可以使用你的引导神力来斥喝那些使用暴力的人。当一个处于你周围30尺内的敌人以一次攻击对一个你之外的生物造成伤害时,以一个反应,你强迫攻击者进行一次感知豁免检定。检定失败则攻击者受到等同于其刚刚造成的伤害的光耀伤害。成功则受到一半伤害。

18级时此光环的距离增加到30尺。

\subsubsection{护卫灵光 Aura of the Guardian}第7级起,你可以以自己的生命为代价保护盟友免受伤害。当一个处于你10尺内的生物受到伤害时,你可以用一个反应动作,魔法性地使你代替该生物承受此伤害。此特性不会传递任何该伤害可能附带的效果,该伤害也无法被任何方式减少。

\subsubsection{守护之魂 Protective Spirit}第15级起,神圣的实体在战斗中为你治疗伤口。若你在战斗中结束你的回合时,剩余的生命值少于一半,且你未处于失能状态,则你恢复等于1d6+你圣武士等级一半的生命值。

\subsubsection{救赎使节 Emissary of Redemption}第20级时,你成为了和平的化身,使你获得下述增益
\begin{itemize}
\item 你对于其他生物造成的所有伤害具有抗力(它们的攻击、法术、和其他效果)。
\item 当一个生物以一次攻击对你造成伤害时,它受到等同于其对你造成的一半伤害的光耀伤害。
\end{itemize}

当你对一个生物攻击、施法或以此特性以外的方式造成伤害,那么在完成一次长休之前此特性对该生物无效。
