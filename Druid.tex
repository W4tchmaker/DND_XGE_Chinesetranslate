%!TEX root = Mainbook.tex
\clearpage
\chapter{德鲁伊 Druid}
\begin{quote}
\emph{即使是在死亡之中,每个生物也都在扮演着它们的角色以维持巨大的平衡。但是现在失衡的因素在增长,一种力量在试图支配自然。这对于凡人种族而言是毁灭性的影响。他们所采取的行为远离了自然,由于他们,更多的腐化即将到来。作为德鲁伊,我们主要通过保护和教育,来维持巨大的平衡,但有些时候我们也不得不奋发图强,对抗危险并将其消灭。——大德鲁伊 Safhran}
\end{quote}

德鲁伊是自然世界的守护者,这也同时意味着他们必须成为自然的声音,说出那些对一般人而言太过于微妙以至于无法听见的真相。许多人因为发现了他们亲近自然的本质而成为了德鲁伊;自然的力量、循环和流动通过惊叹和顿悟填充了他们的内心和灵魂。一些民间的智者和贤者都曾学习过自然,并写下记录它的力量与神秘的卷册,但区别于他们,德鲁伊是一种特殊存在:从某一刻开始,他们能表现出这种自然力量,产生魔法般的现象从而将他们与自然的精魂、与生命的流动联系在了一起。由于他们的这种力量太过于强大和诡秘,德鲁伊通常会被他们周围的人尊敬、回避或者是恐惧。

你的德鲁伊角色或许会成为自然真正的崇敬者,或许是由于蔑视文明而只能从荒野中寻求藉慰。或者你的角色也可以成为一个想要努力将文明世界带入旷野之中,致使它们和谐相处的城市之子。你可以使用以下部分来充实你的德鲁伊,无论你的角色如何进入这个职业之中。

\section{珍视的物品 Treasured Item}

一些德鲁伊会带着一个或更多的圣物,或者是对于他而言有深刻个人意义的物品。这些物品不一定是魔法的,不过一定在概念性或精神性上存有与德鲁伊的内心相联系的重要意义。
当你决定你的角色珍视的物品是什么时,想一想关于它的起源故事:你如何得到这个物品,它为何而言对你重要?
\begin{dndtable}[cX]
\textbf{d6} & \textbf{物品} \\
1 & \emph{矗立在你村子中心的相遇树上的一条枝丫。} \\
2 & \emph{从神圣之河的源头采集的一小瓶河水。} \\
3 & \emph{被捆束在一起的特殊草木。} \\
4 & \emph{一个刻画着动物图案的青铜碗。} \\
5 & \emph{由干葫芦和冬青浆果做成的拨浪鼓。} \\
6 & \emph{由你的导师亲手继承给你的一把小小的金色镰刀。} \\
\end{dndtable}

\section{精神向导 Guiding Aspect}
许多德鲁伊会感觉到与自然中某种具体存在之间强烈的联系,例如一种水体,一种动物,一种树木,或者是一种其他植物。你认同这由你选择的向导;通过它的行为或者是它的本质,它树立了一种使你愿意效仿的榜样。

\begin{dndtable}[cX]
\textbf{d6} & \textbf{精神向导} \\
1 & \emph{紫杉树提醒着你不断更新自己的思想和灵魂,让旧的死去,新的生长。} \\
2 & \emph{橡树象征了力量和活力。在橡树之下冥想让你的身体和心灵重新坚韧且不被动摇。} \\
3 & \emph{河流的永恒流动让你联想到世界的伟大跨度。你提醒自己在行动时要考虑到自然的长远利益。} \\
4 & \emph{海洋仿佛一个持续搅动着混沌力量的锅炉。它令你接受,变化是维持自我在世界中继续存在的必要因素。} \\
5 & \emph{飞鸟是最好的证明,即使最渺小的生物也能存活下来,只要它们远离争端。} \\
6 & \emph{狼的习性展示了个人力量在集群面前的微不足道。} \\
\end{dndtable}
\section{导师 Mentor}
对于未来的德鲁伊而言,寻求一名长老或指导者(或者被发现)来教导它们魔法艺术的基础并非不同寻常。大多数德鲁伊从青年时期就开始跟随导师进行训练,并且导师在塑造学生的态度和信仰时也扮演了重要的角色。

如果你的角色从其他人那儿接受了训练,他会是怎样的个体?你们之间关系的本质是什么?你的导师是否灌输给你一个独特的前景,或者是影响了你选择道路去实现你的目标?

\begin{dndtable}[cX]
\textbf{d6} & \textbf{导师} \\
1 & \emph{你的导师是一个智慧的树人,它在以年为单位的时间跨度中教育着你,而非几天或几个月。} \\
2 & \emph{你被一个树妖所教导,她同时也是一扇通往无底深渊的休眠传送门的守望者。在训练期间,你被托付了从隐藏的威胁中守卫这个世界的责任。} \\
3 & \emph{你的导师总是用猎鹰的形态与你交流。你从来没见过导师的人形样貌。} \\
4 & \emph{你和其他几个年轻人被同一个老德鲁伊教导,直到其中有一个人背叛了你的小组并且杀死了你的导师。} \\
5 & \emph{你的导师只用幻象与你见面。你从来没有见过这个人,你甚至不能确定是否它真的以凡人的形态存在过。} \\
6 & \emph{你的导师是一个熊人,他教导你要对所有活物一视同仁。} \\
\end{dndtable}
\section{德鲁伊结社 Durid Circles}

第2级时,德鲁伊将加入德鲁伊结社。除了在中玩家手册所提到的,下列选项给出了在玩家手册提供的结社以外的额外结社选项。:梦境之环和牧人之环。

\subsection{梦境之环 Circle Of Dreams}

梦境之环的德鲁伊们通常来自与妖精荒野有着很强联系的地区。他们与善良阵营的妖精组成了联盟共通守卫自然世界。这些德鲁伊追求光明和欢乐,并希望它们充满这个世界。他们的魔法能抚平创伤,为沮丧的人带来欢笑;他们守护的土地永远熠熠发光,丰硕富饶,在那里梦境和现实的边界变得模糊,任何疲惫的人都能找到休憩的场所。

\header{梦境之环的特性}
\begin{dndtable}[cX]
\textbf{德鲁伊等级} & \textbf{特性} \\ 
2nd & \emph{\specialcell{仲夏庭院的芬馥 \\Balm of the Summer Court}}\\ 
6th & \emph{\specialcell{月光与阴影的炉心 \\Hearth of Moonlight and Shadow}}\\ 
10th & \emph{隐匿通途 Hidden Paths}\\ 
14th & \emph{梦境行者 Walker in Dreams}\\ 
\end{dndtable}

\subsubsection{仲夏庭院的芬馥 Balm of the Summer Court}
第2级时,你将被灌注来自仲夏庭院的祝福,由此成为缓解旅者乏顿和伤者痛苦的圣洗池。你得到等同于德鲁伊等级的妖精能量池点数,每个点数代表一个d6。

作为一个附赠动作,你可以选择一名在你120尺内的可见盟友,并且花费数量最多等于你德鲁伊等级一半的d6。投掷那些骰子并把结果相加。目标恢复等于其总值的生命值。目标同时也得到等同于所花费骰子数目的临时生命值。

完成一次长休后,你将恢复所有能量池点数。

\subsubsection{月光与阴影的炉心 Hearth of Moonlight and Shadow}
第6级起,无论你在何处设立营灶,那都是你的家园。在进行一次短休或长休时,你可以祈求薄暮庭院的幽影之力来保护你的营地不被外来者侵扰。在休息开始时,你接触空间中的一点,以那一点为中心一个隐形的、半径30尺的球体魔法般地出现,将其下笼罩于全掩护之中。

在这一领域内你和你的盟友在所有感知(观察)检定和敏捷(隐匿)检定中得到+5加值,并且由任何明火放射的光亮(营火,火把以及类似之物)不可见于领域之外。

此效果持续到休息结束或你离开这个领域。

\subsubsection{隐匿通途 Hidden Paths}
第10级起,你可以使用那些被隐藏起来的、无法预测的魔法道路,像妖精一样眨眼间穿越空间。在你的回合,作为一个附赠动作,你可以传送到60尺内一处未被占据的对你可见的区域。或者,你也可以用你的动作来接触一个自愿生物将它传送至30尺内一处未被占据的对你可见的区域。

你可以使用这个能力的次数等于你的感知调整值(至少一次),完成一次长休之后,你恢复所有使用次数。

\subsubsection{梦境行者 Walker in Dreams}
第14级时,妖精荒野的魔力让你能物理性或精神性地穿越梦境空间。

当你完成了一次短休,你可以施展以下法术之一,无需消耗法术位也无需材料成分:\emph{托梦术 Dream}(以你作为信使)、\emph{探知 Scrying}、或\emph{传送法阵 Teleportation Circle}。

以这种方法使用的传送法阵是特别的。取代打开入口通往永久性传送法阵,它将打开的入口通往你在现存位面上完成最后一次长休所在地。若你没有在现存位面上进行过长休,这个法术会失败但不会浪费。

你必须完成一次长休才能再次使用此特性。

\subsection{牧人之环 Circle of the Shepherd}

牧人之环的德鲁伊通晓如何与自然的精魂进行沟通,尤其是野兽之魂与妖精,他们也常常呼唤这些精魂援助自己。这些德鲁伊认为所有的活物都在自然界中发挥作用,因此他们致力于对动物的保护。游牧者们,会在已知范围内尽他们所能地聆听野兽的控诉。他们监视威胁到自然的怪物,斥退进行不必要杀戮的猎人,预防文明侵犯动物栖息地和迁徙所需路径。许多游牧之环的德鲁伊乐意远离城镇,为将时间花费于管理荒野生物而感到充实。

牧人之环的成员通常会因为以下理由进行冒险:对抗那些威胁到他们管理的势力,或者是寻找有助于自己护卫职责的知识和力量。无论这些德鲁伊走到哪里,荒野的精魂都会与他们同行。

\header{牧人之环的特性}
\begin{dndtable}[cX]
\textbf{德鲁伊等级} & \textbf{特性} \\ 
2nd & \emph{\specialcell{林地之语 Speech of the Woods,\\ 精魂图腾 Spirit Totem}}\\ 
6th & \emph{强力召唤 Mighty Summoner}\\ 
10th & \emph{精魂守卫 Guardian Spirit}\\ 
14th & \emph{忠诚召唤 FaithfulSummons}\\ 
\end{dndtable}



\subsubsection{林地之语 Speech of the Woods}
第2级时,你获得了和野兽以及许多精类进行交流的能力。

你获得对于木族语的说、读、和写的能力。此外,野兽能理解你的语言,而你也能理解它们的声音和动作。大多数野兽缺乏足够的智力去表述或理解复杂的概念,但友善的野兽可以复述它在最近听到或看到的事物。此能力并不会让你和野兽缔结特别的友谊,尽管你可以将此技能与讨好它们的礼物结合使用,就像是你能对其他NPC做的那样。


\subsubsection{精魂图腾 Spirit Totem}
第2级起,你获得了招现出自然精魂并借助它们影响你身边的世界的能力。作为一个附赠动作,你可以魔法般地召唤一个虚体的精魂,至处于你周围60尺内、且对你可见的一点。此精魂创造出一道环绕该点的、30尺半径的光环。精魂不算作是生物,也不算作是物体,尽管它有着与对应生物相符的幽魂般的外表。

作为一个附赠动作,你可以移动此精魂至多60尺,至你可见的一点处。

此精魂持续1分钟或直至你失能。你必须完成一次长休或短休才能再次使用此特性。

此精魂的光环效果基于你召唤的精魂的类别,如下列选项所示。

\subparagraph{熊精魂 Bear Spirit}熊精魂授予你和你的伙伴它的力量和坚韧。在精魂出现时,每个处于此光环中、由你选择的生物,获得等于5+你的德鲁伊等级的临时生命值。此外,你和你的盟友在处于光环内时,在力量检定和力量豁免检定中获得优势。

\subparagraph{鹰精魂 Hawk Spirit}鹰精魂是技艺精湛的猎手,用它敏锐的视力标记你的敌人。当一个生物对一个处于此精魂光环内的目标进行一次攻击检定时,你可以使用你的反应让这次攻击检定获得优势。此外,在光环内你和你的盟友得到感知(观察)检定的优势。

\subparagraph{独角兽精魂 Unicorn Spirt}独角兽精魂为周围提供保护。你和你的盟友在用于侦测在此精魂光环内的生物所进行的所有属性检定中获得优势。此外,若你以一个法术位施展了一个为任何处于或不处于光环内的人恢复生命值的法术,每个处于光环内、由你选择的生物也恢复等于你德鲁伊等级的生命值。

\subsubsection{强力召唤 Mighty Summoner}
第6级起,你获得了召唤强力野兽和精类的能力。任何你法术召唤或创造出的野兽或精类获得下述增益:
\begin{itemize}
\item 该生物出现时具有比正常情况下更多的生命值:它每具有一个生命骰将会多出2点额外的生命值。
\item 在克服针对非魔法攻击和伤害的抗性免疫时,来自该生物天生武器的伤害视为是魔法伤害。
\end{itemize}

\subsubsection{精魂守卫 Guardian Spirit}
第10级起,你的精魂图腾会守护那些你用你的魔法呼唤出的野兽和精类。当一个你以一道法术召唤或创造的野兽或精类在你的精魂图腾光环内结束其回合时,该生物恢复等于你德鲁伊等级一半的生命值。

\subsubsection{忠诚召唤 Faithful Summons}

第14级起,你通联的自然精魂会在你最为无助的时候保护你。若你的生命值降到0,或非自愿地变得失能,你可以立即获得以一个9环的法术位施放的\emph{召唤野兽 conjure animals}效益。它召唤出四只由你选择的、挑战等级为2或更低的野兽。被召唤的野兽出现在你周围20尺内。如它们没有从你那里获得命令,它们会保护你免受伤害并攻击你的敌人。此法术无需专注将持续1小时,或直到你解消它(无需动作)。

你必须完成一次长休才能再次使用此特性。

\section{学习野兽形态 Learning Beast Shapes}
在玩家手册中荒野形态职业能力能让你变形为过去曾见过的一种野兽。这个规则给你了惊人的灵活性,使你能轻而易举地堆积野兽形态作为自己变形的选择,但是你必须遵循那本书中的野兽形态表的限制。

当你在德鲁伊的第二个等级得到荒野形态能力时,你或许想要知道哪些野兽你已经见过。以下列表是从怪物手册中修订的,在不同环境中最有可能遇见的生物。考虑你的德鲁伊成长的环境,随后在你的德鲁伊到达2级时选择适当你的德鲁伊可能见过的动物表格。

这个表格也能帮助你和你的DM判断你或许会在旅行中见过哪些生物。此外,这个表格也包含了每个野兽的挑战等级,记录了他们的飞行或游泳速度。这些信息会帮助你决定你是否适任该野兽形态。

这些表格包括了对于荒野形态(至多CR1)或者是月亮结社的结社形态(至多CR6)而言所有合适的独特野兽。


\header{寒带}
\begin{dndtable}[c p{3cm} p{3.5cm}]
\textbf{CR} & \textbf{野兽} & \textbf{飞行/游泳} \\
0 & \emph{猫头鹰} & \emph{飞行} \\
1/8 & \emph{血鹰} & \emph{飞行} \\
1/4 & \emph{巨猫头鹰} & \emph{飞行} \\
1 & \emph{棕熊} & \emph{——} \\
2 & \emph{白熊} & \emph{游泳} \\
2 & \emph{剑齿虎} & \emph{——} \\
6 & \emph{猛犸象} & \emph{——} \\
\end{dndtable}
\vspace*{-1cm}
\header{海岸}
\begin{dndtable}[c p{3cm} p{3.5cm}]
\textbf{CR} & \textbf{野兽} & \textbf{飞行/游泳} \\
0 & \emph{螃蟹} & \emph{游泳} \\
0 & \emph{鹰} & \emph{飞行} \\
1/8 & \emph{血鹰} & \emph{飞行} \\
1/8 & \emph{巨螃蟹} & \emph{游泳} \\
1/8 & \emph{毒蛇} & \emph{游泳} \\
1/8 & \emph{蚊蝠} & \emph{飞行} \\
1/4 & \emph{巨蜥蜴} & \emph{——} \\
1/4 & \emph{巨狼蛛} & \emph{——} \\
1/4 & \emph{翼龙} & \emph{飞行} \\
1 & \emph{巨鹰} & \emph{飞行} \\
1 & \emph{巨蟾蜍} & \emph{游泳} \\
2 & \emph{蛇颈龙} & \emph{游泳} \\
\end{dndtable}
\header{荒漠}
\begin{dndtable}[c p{3cm} p{3.5cm}]
\textbf{CR} & \textbf{野兽} & \textbf{飞行/游泳} \\
0 & \emph{猫} & \emph{——} \\
0 & \emph{鬣狗} & \emph{——} \\
0 & \emph{豺狗} & \emph{——} \\
0 & \emph{蝎子} & \emph{——} \\
0 & \emph{秃鹫} & \emph{飞行} \\
1/8 & \emph{骆驼} & \emph{——} \\
1/8 & \emph{飞蛇} & \emph{飞行} \\
1/8 & \emph{骡子} & \emph{——} \\
1/8 & \emph{毒蛇} & \emph{游泳} \\
1/8 & \emph{蚊蝠} & \emph{飞行} \\
1/4 & \emph{蟒蛇} & \emph{游泳} \\
1/4 & \emph{巨蜥蜴} & \emph{——} \\
1/4 & \emph{巨毒蛇} & \emph{游泳} \\
1/4 & \emph{巨狼蛛} & \emph{——} \\
1 & \emph{巨鬣狗} & \emph{——} \\
1 & \emph{巨蜘蛛} & \emph{——} \\
1 & \emph{巨蟾蜍} & \emph{游泳} \\
1 & \emph{巨秃鹫} & \emph{飞行} \\
1 & \emph{狮子} & \emph{——} \\
2 & \emph{巨蟒蛇} & \emph{游泳} \\
3 & \emph{巨蝎} & \emph{——} \\
\end{dndtable}
\header{\emph{森林}}
\begin{dndtable}[c p{3cm} p{3.5cm}]
\textbf{CR} & \textbf{野兽} & \textbf{飞行/游泳} \\
0 & \emph{狒狒} & \emph{——} \\
0 & \emph{獾} & \emph{——} \\
0 & \emph{猫} & \emph{——} \\
0 & \emph{鹿} & \emph{——} \\
0 & \emph{鬣狗} & \emph{——} \\
0 & \emph{猫头鹰} & \emph{飞行} \\
1/8 & \emph{血鹰} & \emph{飞行} \\
1/8 & \emph{飞蛇} & \emph{飞行} \\
1/8 & \emph{巨鼠} & \emph{——} \\
1/8 & \emph{巨鼬} & \emph{——} \\
1/8 & \emph{毒蛇} & \emph{游泳} \\
1/8 & \emph{獒犬} & \emph{——} \\
1/8 & \emph{蚊蝠} & \emph{飞行} \\
1/4 & \emph{野猪} & \emph{——} \\
1/4 & \emph{蟒蛇} & \emph{游泳} \\
1/4 & \emph{麋鹿} & \emph{——} \\
1/4 & \emph{巨獾} & \emph{——} \\
1/4 & \emph{巨蝙蝠} & \emph{飞行} \\
1/4 & \emph{巨蛙} & \emph{游泳} \\
1/4 & \emph{巨蜥蜴} & \emph{——} \\
1/4 & \emph{巨猫头鹰} & \emph{飞行} \\
1/4 & \emph{巨毒蛇} & \emph{游泳} \\
1/4 & \emph{巨狼蛛} & \emph{——} \\
1/4 & \emph{黑豹} & \emph{——} \\
1/4 & \emph{狼} & \emph{——} \\
1/2 & \emph{猿} & \emph{——} \\
1/2 & \emph{黑熊} & \emph{——} \\
1/2 & \emph{巨黄蜂} & \emph{飞行} \\
1 & \emph{棕熊} & \emph{——} \\
1 & \emph{恐狼} & \emph{——} \\
1 & \emph{巨鬣狗} & \emph{——} \\
1 & \emph{巨蜘蛛} & \emph{——} \\
1 & \emph{巨蟾蜍} & \emph{游泳} \\
1 & \emph{老虎} & \emph{——} \\
2 & \emph{巨野猪} & \emph{——} \\
2 & \emph{巨蟒} & \emph{游泳} \\
2 & \emph{巨麋鹿} & \emph{——} \\
\end{dndtable}

\header{草原}
\begin{dndtable}[c p{3cm} p{3.5cm}]
\textbf{CR} & \textbf{野兽} & \textbf{飞行/游泳} \\
0 & \emph{猫} & \emph{——} \\
0 & \emph{鹿} & \emph{——} \\
0 & \emph{鹰} & \emph{飞行} \\
0 & \emph{山羊} & \emph{——} \\
0 & \emph{鬣狗} & \emph{——} \\
0 & \emph{豺狗} & \emph{——} \\
0 & \emph{秃鹫} & \emph{飞行} \\
1/8 & \emph{血鹰} & \emph{飞行} \\
1/8 & \emph{飞蛇} & \emph{飞行} \\
1/8 & \emph{巨鼬} & \emph{——} \\
1/8 & \emph{毒蛇} & \emph{游泳} \\
1/8 & \emph{蚊蝠} & \emph{飞行} \\
1/4 & \emph{斧嘴鸟} & \emph{——} \\
1/4 & \emph{野猪} & \emph{——} \\
1/4 & \emph{麋鹿} & \emph{——} \\
1/4 & \emph{巨毒蛇} & \emph{游泳} \\
1/4 & \emph{巨狼蛛} & \emph{——} \\
1/4 & \emph{黑豹(金钱豹)} & \emph{——} \\
1/4 & \emph{翼龙} & \emph{飞行} \\
1/4 & \emph{骑乘马} & \emph{——} \\
1/4 & \emph{狼} & \emph{——} \\
1/2 & \emph{巨山羊} & \emph{——} \\
1/2 & \emph{巨黄蜂} & \emph{飞行} \\
1 & \emph{巨鹰} & \emph{飞行} \\
1 & \emph{巨鬣狗} & \emph{——} \\
1 & \emph{巨秃鹫} & \emph{飞行} \\
1 & \emph{狮子} & \emph{——} \\
1 & \emph{老虎} & \emph{——} \\
2 & \emph{异龙} & \emph{——} \\
2 & \emph{巨野猪} & \emph{——} \\
2 & \emph{巨麋鹿} & \emph{——} \\
2 & \emph{犀牛} & \emph{——} \\
3 & \emph{甲龙} & \emph{——} \\
4 & \emph{象} & \emph{——} \\
5 & \emph{三角龙} & \emph{——} \\
\end{dndtable}

\header{丘陵}
\begin{dndtable}[c p{3cm} p{3.5cm}]
\textbf{CR} & \textbf{野兽} & \textbf{飞行/游泳} \\
0 & \emph{狒狒} & \emph{——} \\
0 & \emph{鹰} & \emph{飞行} \\
0 & \emph{山羊} & \emph{——} \\
0 & \emph{鬣狗} & \emph{——} \\
0 & \emph{渡鸦} & \emph{飞行} \\
0 & \emph{秃鹫} & \emph{飞行} \\
1/8 & \emph{血鹰} & \emph{飞行} \\
1/8 & \emph{巨鼬} & \emph{——} \\
1/8 & \emph{獒犬} & \emph{——} \\
1/8 & \emph{骡子} & \emph{——} \\
1/8 & \emph{毒蛇} & \emph{游泳} \\
1/8 & \emph{蚊蝠} & \emph{飞行} \\
1/4 & \emph{斧嘴鸟} & \emph{——} \\
1/4 & \emph{野猪} & \emph{——} \\
1/4 & \emph{麋鹿} & \emph{——} \\
1/4 & \emph{巨猫头鹰} & \emph{飞行} \\
1/4 & \emph{巨狼蛛} & \emph{——} \\
1/4 & \emph{黑豹(美洲狮)} & \emph{——} \\
1/4 & \emph{狼} & \emph{——} \\
1/2 & \emph{巨山羊} & \emph{——} \\
1 & \emph{棕熊} & \emph{——} \\
1 & \emph{恐狼} & \emph{——} \\
1 & \emph{巨鹰} & \emph{飞行} \\
1 & \emph{巨鬣狗} & \emph{——} \\
1 & \emph{狮子} & \emph{——} \\
2 & \emph{异龙} & \emph{——} \\
2 & \emph{巨野猪} & \emph{——} \\
2 & \emph{巨麋鹿} & \emph{——} \\
\end{dndtable}
\header{山地}
\begin{dndtable}[c p{3cm} p{3.5cm}]
\textbf{CR} & \textbf{野兽} & \textbf{飞行/游泳} \\
0 & \emph{鹰} & \emph{飞行} \\
0 & \emph{山羊} & \emph{——} \\
1/8 & \emph{血鹰} & \emph{飞行} \\
1/8 & \emph{蚊蝠} & \emph{飞行} \\
1/4 & \emph{翼龙} & \emph{飞行} \\
1/2 & \emph{巨山羊} & \emph{——} \\
1 & \emph{巨鹰} & \emph{飞行} \\
1 & \emph{狮子} & \emph{——} \\
2 & \emph{巨麋鹿} & \emph{——} \\
2 & \emph{剑齿虎} & \emph{——} \\
\end{dndtable}

\header{沼泽}
\begin{dndtable}[c p{3cm} p{3.5cm}]
\textbf{CR} & \textbf{野兽} & \textbf{飞行/游泳} \\
0 & \emph{老鼠} & \emph{——} \\
0 & \emph{渡鸦} & \emph{飞行} \\
1/8 & \emph{巨老鼠} & \emph{——} \\
1/8 & \emph{毒蛇} & \emph{游泳} \\
1/8 & \emph{蚊蝠} & \emph{飞行} \\
1/4 & \emph{蟒蛇} & \emph{游泳} \\
1/4 & \emph{巨蛙} & \emph{游泳} \\
1/4 & \emph{巨蜥蜴} & \emph{——} \\
1/4 & \emph{巨毒蛇} & \emph{游泳} \\
1/2 & \emph{鳄鱼} & \emph{游泳} \\
1 & \emph{巨蜘蛛} & \emph{——} \\
1 & \emph{巨蟾蜍} & \emph{游泳} \\
2 & \emph{巨蟒蛇} & \emph{游泳} \\
5 & \emph{巨鳄} & \emph{游泳} \\
\end{dndtable}

\header{地底}
\begin{dndtable}[c p{3cm} p{3.5cm}]
\textbf{CR} & \textbf{野兽} & \textbf{飞行/游泳} \\
0 & \emph{巨火甲虫} & \emph{——} \\
1/8 & \emph{巨老鼠} & \emph{——} \\
1/8 & \emph{蚊蝠} & \emph{飞行} \\
1/4 & \emph{巨蝙蝠} & \emph{飞行} \\
1/4 & \emph{巨蜈蚣} & \emph{——} \\
1/4 & \emph{巨蜥蜴} & \emph{——} \\
1/4 & \emph{巨毒蛇} & \emph{游泳} \\
1/2 & \emph{鳄鱼} & \emph{游泳} \\
1 & \emph{巨蜘蛛} & \emph{——} \\
1 & \emph{巨蟾蜍} & \emph{游泳} \\
2 & \emph{巨蟒蛇} & \emph{游泳} \\
2 & \emph{白熊(穴居熊)} & \emph{游泳} \\
\end{dndtable}

\header{水下}
\begin{dndtable}[c p{3cm} p{3.5cm}]
\textbf{CR} & \textbf{野兽} & \textbf{飞行/游泳} \\
0 & \emph{食人鱼} & \emph{游泳} \\
1/4 & \emph{蟒蛇} & \emph{游泳} \\
1/2 & \emph{巨海马} & \emph{游泳} \\
1/2 & \emph{礁鲨} & \emph{游泳} \\
1 & \emph{巨章鱼} & \emph{游泳} \\
2 & \emph{巨蟒蛇} & \emph{游泳} \\
2 & \emph{寻猎鲨} & \emph{游泳} \\
2 & \emph{蛇颈龙} & \emph{游泳} \\
3 & \emph{虎鲸} & \emph{游泳} \\
5 & \emph{巨鲨} & \emph{游泳} \\
\end{dndtable}