%!TEX root = Mainbook.tex
\clearpage
\chapter{术士 Sorcerer}
\begin{quote}
\emph{二流的魔法才需要锻炼,真正的力量则与生俱来。
——Hennet 提亚马特后裔}
\end{quote}

若是谈及在需要时汲取自己的能力,相比起其他角色,术士要容易得多。他们的力量不仅源于自身,反倒是抑制起来需要好些精力。除了极少数天选之子,术士就是生来的术士。和其他那些必须学习,精进,利用才能的角色不同,术士的力量定义了术士。

天生的魔法存在走来走去会让很多人坐立不安,因此术士的身份本身就容易滋生不信任并带来他人的怀疑。尽管如此,许多术士还是通过为那些魔法天赋不及他们的同胞贡献提供些帮助,成功克服了这种偏见。

力量觉醒时周遭的态度,通常也极大地影响着术士。有些人将其视作值得期待的天赋,它的出现自然值得在他们间庆祝一番。而也有些术士则被视作瘟神,力量以一种可怖的方式降临后,他们转瞬间就被逐出家门。

扮演术士挺有挑战性,却也物有所值。下列内容将提供些建议,助你扮演并深化术士角色。

\section{神秘起源 Arcane Origin}一些术士知道他们的力量源自何处,基于他们的能力如何展现。其他术士只能推测,因他们的力量以不固定的原因及形式展现。

你的角色知道你魔法力量的源头吗?它是否与某个遥远的亲戚,一个宇宙事件,或者是某种机缘巧合联系在一起?如果你的术士不知道他的力量是从哪来的,你的DM可以使用以下表格(或选一个起源),并在冒险中向你揭示它。

\begin{dndtable}[cX]
\textbf{d6} & \textbf{起源} \\
1 & \emph{你的力量来自于你家族的血统。你与某些强大的生物有关,或你继承了祝福或诅咒。} \\
2 & \emph{你是另一个存在的本位面的化身。} \\
3 & \emph{一个强大的实体进入了这个世界。它的魔力改变了你。} \\
4 & \emph{你的出生被古代的经文所预言,它预言了你的力量将被用来达到可怕的目的。} \\
5 & \emph{你是几代人精心培育的产物。} \\
6 & \emph{你是在一个炼金术士的大桶里做成的。} \\
\end{dndtable}

\section{反应 Reaction}当一个新的术士降世,无论是其出生的时,还是当他的力量变得明显时,结果很大程度上取决于旁观者对他们所看到的事物的反应。

当你的魔法力量出现时,周围的世界是又有何反应的?旁人是支持,害怕,还是介于两者之间?
\begin{dndtable}[cX]
\textbf{d6} & \textbf{反应} \\
1 & \emph{你的力量被你周围的人视为巨大的祝福,你也被期望用它们来为身边人服务。} \\
2 & \emph{你的力量在他们变得明显的时候造成了毁灭乃至死亡,你被当作罪犯对待。} \\
3 & \emph{你的邻居憎恨你,害怕你的力量,他们离你躲得远远的。} \\
4 & \emph{你注意到一个邪恶的邪教组织,他们在计划利用你的能力。} \\
5 & \emph{你周围的人都认为你的力量是对你的家庭的一种诅咒,因为过去的罪行。} \\
6 & \emph{据传,你的力量与一个古老的疯王、在一个世纪前的血腥起义中有某些关联。} \\
\end{dndtable}


\section{超自然标记 Supernatural Mark}休息时的术士几乎与旁人无异,只有当他们采用魔法时才会暴露自己的本性。尽管如此,很多术士会有隐约可见却又着实存在的物理特征,来把他们与常人区分开来。
如果你的术士有个超自然标记,也许很容易就能遮掩住,也可能你时常将其骄傲的展示出来。
\begin{dndtable}[cX]
\textbf{d6} & \textbf{超自然印记} \\
1 & \emph{你眼睛的颜色不同寻常,比如红色。} \\
2 & \emph{你一只脚上有个额外的脚趾。} \\
3 & \emph{你的一只耳朵明显比另一只大。} \\
4 & \emph{你的头发长得惊人。} \\
5 & \emph{当你咀嚼的时候,你会不断地皱起鼻子。} \\
6 & \emph{你的脖子每天都会出现一个红色斑点,但一小时后就会消失。} \\
\end{dndtable}

\section{术士征兆 Sign Of Sorcery}正如世人所知,一些术士能更好的控制住自己施放的法术。有时术士在施法时,其身上的魔力径流几近暴走。但是,即使当一个术士的法术施放的稳稳当当,施法的迹象可能也会很明显,能让人清清楚楚地判断这魔法的能量究竟从何而来。

当你的术士角色施法时,会显示出魔法征兆吗?这个征兆是和你的法力之源有关,和你的其他方面有关,还是看似完全随机?

\begin{dndtable}[cX]
\textbf{d6} & \textbf{征兆} \\
1 & \emph{你吟唱咒文时声如泰坦。} \\
2 & \emph{你施法时那一瞬,周围的空间黑暗阴郁。} \\
3 & \emph{你在施法时和其后几秒间汗流如注。} \\
4 & \emph{无论你何时施法,头发和衣角都有如微风轻拂般飘动。} \\
5 & \emph{若站着施法,你会漂浮至6英寸高的空中,然后轻轻落回地面。} \\
6 & \emph{在你施法时,蓝色的虚无火焰会在你头上出现,然后赫然消失} \\
\end{dndtable}

\section{术法起源 Sorcerous Origins}
第3级时,术士将获得其术法起源特性,下列选项给出了在玩家手册提供的术法起源以外的额外术法起源选项:神圣之魂、暗影魔法和风暴巫术。

\subsection{神性之魂 Divine Soul}有时,支持着术士的魔力火花来自他灵魂中闪耀着的一点神力。拥有这样一种神迹的灵魂,表明你内在的魔法可能来自于一个遥远而强大的家族与某种神圣的存在之间有所联系。也许你的神是一个天使,变成了一个凡人,奉神的名义来战斗。或者你的出生可能与一个古老的预言相符,昭示着你是神的仆人,或者是神圣魔法的容器。

一个具有自然魅力的神性灵魂,被一些教派视为威胁。作为一个指挥着神圣的力量的局外人,神性的灵魂可以通过声称与神的直接联系,来破坏现有的秩序。

在某些文化中,只有那些具备神性灵魂之力的人,才能统御宗教势力。在这些国度,教会的权利由几条血脉支配着,代代相传。

\begin{dndtable}[cX]
\textbf{等级} & \textbf{特性} \\
1st & \emph{\specialcell{神性魔法 Divine Magic,\\众神眷恩 Favored by the Gods}} \\
6th & \emph{强效治疗 Empowered Healing} \\
14th & \emph{异界之翼 Otherworldly Wings} \\
18th & \emph{神秘复原 Unearthly recovery} \\
\end{dndtable}

\subsubsection{神性魔法 Divine Magic}你对于神力的联系允许你学习通常与牧师职业相关的法术。当你的施法特性让你学会一个术士戏法或一个1级或更高级的术士法术时,在术士法术列表之外,你可以从牧师法术列表中选择新法术。除此之外,你必须满足选择该法术的所有限制,而它对你而言成为一个术士法术。

此外,选择一种对你神力来源的亲和力:善、恶、守序、混乱或中立。你可以根据这个关联,学习一个额外的法术,如下所示。这对你来说是一个术士法术,但它不与你术士法术的数量有关。如果你以后替换了这个法术,你必须用牧师的法术列表来替换它。
\begin{dndtable}[cX]
\textbf{亲和} & \textbf{法术} \\
\emph{善良} & \emph{疗伤术 cure wounds} \\
\emph{邪恶} & \emph{致伤术 inflict wounds} \\
\emph{守序} & \emph{祝福术 bless} \\
\emph{混乱} & \emph{灾祸术 bane} \\
\emph{中立} & \emph{防护善恶 protectionfrom evil and good} \\
\end{dndtable}

\subsubsection{众神眷恩 Favored by the Gods}第1级起,神圣力量指引着你的命运。若你未通过一次豁免检定或在一次攻击检定中失手,你可以投2d4,并将它加到总值上,这可能会改变结果。

你必须完成一次短休长休才能再次使用此特性。

\subsubsection{强效治疗 Empowered Healing}6级起,你可以通过神圣之力增幅治疗法术。当你或你的盟友在你周围5尺内掷骰来决定法术恢复生命值的时候,只要你没有失能,你便可以花费1个术法点,重扔其中任意数量的骰子。每轮只能使用一次该能力。

\subsubsection{异界之翼 Otherworldly Wings}14级起,你可以使用一个附赠动作,在你的背部展现一对光翼。当光翼展现时,你具有30尺的飞行速度。它们会一直展现着,直到你失能、死亡、或者你以一个附赠动作遣散。

你为你神性魔法的特性所选择的亲和力,决定了光翼的外观:鹰翼属于善良或守序,蝠翼属于邪恶或混乱,中立则是蜻蜓的翅膀。

\subsubsection{神秘复原 Unearthly recovery}
18级起,你具备了克服重伤的能力。当你的hp不足一半时,以一个附赠动作,你可以恢复等同于你一半生命值的生命。

你必须完成一次长休才能再次使用此特性。

\subsection{幽影魔法 Shadow Magic}你是一个幽影的生物, 你内在的魔力来自堕影冥界本身。 你可以把你的世系追溯到那个地方实存的某处, 又或者你将自己暴露在了它的能量之下并被其所改变。暗影魔法的力量给你的肉身蒙上了一层怪异的柩衣。维持你生命的火花被抑制,犹如它为了存留而挣扎着与灌输了你灵魂的黑暗力量对抗。在你的选择下,你可以从下列幽影术士异闻表中自选或投掷出属于自己角色的特点。

\begin{dndtable}[cX]
\textbf{d6} & \textbf{异闻} \\
1 & \emph{你摸起来一直是冰冷的。} \\
2 & \emph{当你熟睡时看上去没有呼吸。(尽管你需要呼吸来维生。)} \\
3 & \emph{你很少流血,即使受了重伤。} \\
4 & \emph{你的心脏一分钟才脉动一下,有时候你自己都感到吃惊。} \\
5 & \emph{你难以记住生物和尸体应该被区别对待。} \\
6 & \emph{你上周只眨了一次眼。} \\
\end{dndtable}

\begin{dndtable}[cX]
\textbf{等级} & \textbf{特性} \\
1st & \emph{\specialcell{幽暗之瞳 Eyes of the Dark,\\终焉之力 Stength of the Grave}} \\
3rd & \emph{幽暗之瞳 Eyes of the Dark(黑暗术)} \\
6th & \emph{凶兆猎犬 Hound of Ill Omen} \\
14th & \emph{幽影漫步 Shadow Walk} \\
18th & \emph{幽暗形态 Umbral Form} \\
\end{dndtable}

\subsubsection{幽暗之瞳 Eyes of the Dark}1级起你获得120尺黑暗视觉。

当你在该职业3级时,你掌握了\emph{黑暗术 darkness},这不会占用你术士已知法术数量。你可以用2点术法点或1个法术位来释放。当你用法术点施法时,你能够看穿该法术所制造的黑暗。

\subsubsection{终焉之力 Stength of the Grave}1级起,存在于生和死的朦胧状态中使得你难以被击败。当生命值被降至0点时,你可以通过一个DC=该次所受伤害+5的魅力豁免。若豁免成功,作为代替你的生命值被降至1点。若该次伤害为光耀或重击时你不可使用此能力。

通过豁免后,你必须完成一次长休才能再次使用此特性。

\subsubsection{凶兆猎犬 Hound of Ill Omen}6级时,你获得一种能力,你能够驱使嚎叫的幽暗生物来折磨你的敌人的。你可以以一个附赠动作消耗3点术法点选择一个120尺内你可见的生物作为目标,用魔法召唤一头凶兆猎犬的猎犬去猎杀它。凶兆猎犬用恐狼的模板但有以下变化:

\begin{itemize}
\item 猎犬的体型由大体型变为中等体型,猎犬视作怪兽而不是野兽。
\item 当猎犬被召唤时获得等同于你术士等级一半的临时生命值。
\item 猎犬可以像穿越困难地形一样穿过其他生物或物体。如果当猎犬的回合结束时处于某个物体内则会受到5点的力场伤害。
\item 当猎犬的回合开始时,自动知晓猎物的所在。如果目标处于躲藏状态,它对猎犬来说不再处于该状态。
\end{itemize}

猎犬被召唤时出现在目标30尺内一个由你选择的空间。为猎犬投掷先攻。当猎犬行动时,它只能以最短路径冲向目标猎物,以及只能在攻击目标猎物时执行动作。猎犬可以使用借机攻击,但对象只能是目标猎物。此外,当凶兆猎犬在目标猎物5尺内时,目标对你所释放的任何法术的豁免上都承受劣势。凶兆猎犬会在它自己降为0血、猎物降为0血或5分钟后消失。

\subsubsection{幽影漫步 Shadow Walk}14级时,你获得在暗影间穿梭的能力。当你处于微光或黑暗中时,以一个附赠动作你可以魔法地将自己传送至120尺内另一处未被占据并自己可见的微光或黑暗中。

\subsubsection{幽暗形态 Umbral Form}18级时,你可以耗费6点术法点以一个附赠动作魔法地将自己转变为暗影形态。在此形态中,你对除力场和光耀之外的所有伤害形势都具有抗性。并且你可以像穿越困难地形一般地穿过其他生物或物体。若回合结束时你处于某个物体内则会受到5点的力场伤害。

你能维持这种状态1分钟。当你以一个附赠动作来解除暗影形态、处于失能或死亡时,此效应中止。


\subsection{狂怒风暴 Storm Sorcery}你内在的魔法源于风元素之力量。也许你在狂风呼啸中诞生,而那狂风因其威力之强而仍被市井传颂;也许你的血脉受到强大风元素生物的影响,比如风元素大公,又或风巨灵。不论情况如何,风暴之魔力早已浸透你的灵魂。

风暴术士是极有价值的海员。其魔法足以驾驭其周边的风与气象。他们的能力亦已被证明在对抗沙华鱼人或海盗等水上威胁时极为有效。

\begin{dndtable}[cX]
\textbf{等级} & \textbf{特性} \\
1st & \emph{\specialcell{风语者 Wind Speaker,\\暴风魔法 Tempestuous Magic}} \\
6th & \emph{\specialcell{风暴导向 Storm Guide, \\风暴导向 Storm Guide}} \\
14th & \emph{暴风狂怒 Storm's Fury} \\
18th & \emph{清风之灵 Wind Soul} \\
\end{dndtable}

\subsubsection{风语者 Wind Speaker}第1级起,你操纵的奥术魔法浸染着风元素之力。你可以听、说、读、写原初语(Primordial)。(了解这种语言允许你了解这些说这种语言的种族,同时你说的话也被他们理解:如水族语Aquan,气族语Auran,火族语Ignan,土族语Terran)。

\subsubsection{暴风魔法 Tempestuous Magic}第1级起,你与风元素魔法紧密联系起来。每当你在你自己的行动轮内施展一个并非戏法的法术时,你可以用附赠动作导致旋转的风元素飙风围绕着你。此时你可以飞行10尺,而不引起借机攻击。

\subsubsection{风暴之心 Heart of the Storm}第6级起,你对闪电和雷鸣伤害获得抗力。每当你施展一个1环或以上且造成闪电或雷鸣伤害的法术时,猛烈的魔法从你身体内爆发而出。这次爆发导致所有10尺内你选择的可见生物(每次使用此能力都指定),都承受等于你术士等级一半的闪电或雷鸣伤害。

\subsubsection{风暴导向 Storm Guide}从6级起,你获得对你周边气象的微妙掌控。

如果正在下雨,你可以用一个动作,使你身边20尺半径的范围停止下雨。你可以以一个附赠动作终止这一效果。

如果正在刮风,你可以每轮以一个附赠动作掌控你身边一百尺半径范围的风向。风会保持吹往你指定的方向,直到你的下一个行动轮。你无法改变风速。

\subsubsection{暴风狂怒 Storm's Fury}第14级起,每当你在一次近战攻击中被命中,你可以用你的反应对攻击者造成等于你术士等级的闪电伤害。此外,攻击者必须进行一次对抗你法术豁免DC的力量检定。一旦豁免失败,攻击者将被直线推开20尺。

\subsubsection{清风之灵 Wind Soul}18级时,你获得60尺飞行速度,且免疫闪电与雷鸣伤害。

作为一个动作,你可以将你的飞行速度降低到30尺,持续一小时。当你这么做时,你可以指定在你身边30尺内的,3+你的魅力调整值个生物。这些被选定的生物获得30尺飞行速度,持续1小时。你必须完成一次短休或长休才能再次使用此特性。