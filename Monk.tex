%!TEX root = Mainbook.tex
\clearpage

\chapter{武僧 Monk}
\begin{quote}
\emph{勿将吾之沉静看做对汝恶行之领受。汝咆哮威胁之刻,吾已计划好四种途径用吾之双手折断汝之颈脖。——花卉宗师 Ember }
\end{quote}

武僧们走的是一条矛盾的道路。他们如同法师般研习他们的技艺,也如同法师般不着护具,通常不使用武器。但尽管如此,他们仍是致命的武士,他们的能力与狂暴的野蛮人或训练有素的战士不相上下。武僧们接受这种存于表象的矛盾,因为它是所有修道研究的核心。通过完全了解自身,从而去了解更加广阔的世界。

一名武僧专注于自我内心的掌控,这让他们大多脱离了社会,且更关心他们的个人经历而非其他地方发生的事情。参与到冒险当中的武僧在英雄中是稀有且罕见的,他们为了追求完美而越过了修道院的围墙,进入了俗世之中。

扮演一个武僧提供了许多有趣的机会去尝试不同的东西。为了进一步区分你的武僧角色,你可以考虑如下小节中的选项。

\section{修道院 Monastery}一个在修道院学习的武僧通常是在为修道者的禁欲生活作出准备。大部分进入修道院的人都将其作为他们余生的家,除了那些冒险者和其他有理由离开的人。对于这些人来说,修道院可以成为旅行之间的避所,或在其需要的时候作为援助来源。

你的修道院是怎么样的,它位于哪里?参与它会让你的角色经历变得独特或与众不同吗?

\begin{dndtable}[cX]
\textbf{d6} & \textbf{修道院} \\
1 & \emph{你的修道院是从山腰上被雕刻成型的,它位于一个若隐若现的危险道路。} \\
2 & \emph{你的修道院在一片浩瀚无边的参天大树群的树枝高处,它位于妖精荒野。} \\
3 & \emph{你的修道院很早以前由一个云巨人建立,它位于一座只能通过飞行才能抵达的云城堡内。} \\
4 & \emph{你的修道院建在温泉,间歇泉和硫磺池的火山旁边。你经常会收到来自火矮人贸易商的访问。} \\
5 & \emph{你的修道院是由侏儒建立的,如地下迷宫般充斥着大量隧道和房间。} \\
6 & \emph{你的修道院是在冰山之上,从冰块之中雕琢而出的。} \\
\end{dndtable}

\section{修道院标识 Monastic Icon}尽管武僧的生活方式中避开了物质主义和个人财产,但修道院标识在维持戒律和确认身份时仍然起到重要的作用。一些修道院戒律会对某些动物给予特殊的重视,这往往是因为这个动物与其历史有所联系,或者是因为它是武僧们试图效仿的品质的典范。

如果你的角色的修道院有一个特殊的戒律,你可能会在衣服上不显眼的地方穿戴其粗糙的图案,作为识别标记。也有可能你的标识图标没有物理形式,但你可以通过手势或姿势表达,而其他的武僧也许会知道其作何解释。

\begin{dndtable}[cX]
\textbf{d6} & \textbf{修道院标识} \\
1 & \emph{猴子。你的戒律欣赏猴子的快速反应及穿越林木的迅捷。} \\
2 & \emph{龙龟。你的修道院位置在海边,他们通过背诵古代的祈祷和提供鲜花花环来纪念龙龟,这种海洋中生活的精魂。} \\
3 & \emph{麒麟。你的修道院认为他们存在的意义是如同麒麟一样守望和保护这片土地。} \\
4 & \emph{枭熊。 你的修道院敬仰一群枭熊家族,并已与它们一起共存了好几个世代。} \\
5 & \emph{九头蛇。你的戒律向往九头蛇那样在瞬间打出复数次攻击的能力。} \\
6 & \emph{龙。一条龙曾存在于你的修道院中,它所带来的影响在它离开之后仍然长久。} \\
\end{dndtable}

\section{大师 Master}你的学习很有可能是在一位向你传授戒律的大师指导下进行的。大师塑造你对武术的理解和你对此世的态度。你的大师是位什么样的人?你和大师的关系又如何影响了你?
\begin{dndtable}[cX]
\textbf{d6} & \textbf{大师} \\
1 & \emph{你的大师是一个专横的人,你必须在单人战斗中击败他才算是完成他对你的授业。} \\
2 & \emph{你的大师很和善,他教导你去追求和平的事业。} \\
3 & \emph{你的大师总是严厉地将你逼向极限。在一次极其残酷的试炼中你差点失去你的一只眼睛。} \\
4 & \emph{你的大师在教导你时看起来很善良,但最后却背叛了你的修道院。} \\
5 & \emph{你的大师总是对你冷漠而疏远。你经常怀疑你俩之间的关系。} \\
6 & \emph{你的大师是善良和慷慨的,从不挑剔你的进步。然而你觉得你从来没有满足他对你的期望。} \\
\end{dndtable}


\section{武僧宗派 Monastic Tradtions}
第3级时,武僧将获得其武僧宗派,下列选项给出了在玩家手册提供的宗派以外的额外宗派选项:醉拳宗,剑圣宗和日魂宗。

\subsection{醉拳宗 Way of the Drunken Master}醉拳宗的导师教导其学徒跟随醉汉古怪且不可预测的行为来移动。一个醉拳大师总是摇摇晃晃、蹒跚摇摆着他的脚,战斗方式看似蹩脚而无力。然而他们飘忽不定的跌撞下,掩盖的是阻碍,招架,推进,攻击和撤退嵌凑成的精妙连击。

醉拳大师喜欢装疯卖傻,抚慰哀者,恭顺倨者,但当战斗的号角打响,醉拳宗的大师会摇身一变成为可怕而可敬的对手。

\begin{dndtable}[cX]
\textbf{等级} & \textbf{特性} \\
3rd & \emph{\specialcell{习熟练 Bonus Proficiencies,\\醉拳窍 Drunken Technique}}\\
6th & \emph{酣荡步 Tipsy Sway}\\
11th & \emph{酩酊运 Drunkard's Luck}\\
17th & \emph{为酒狂 Intoxicated Frenzy}\\
\end{dndtable}

\subsubsection{习熟练 Bonus Proficiencies}
第3级时,若你还没有获得表演技能项熟练,则你获得该熟练橡。你的武术技巧将战斗与舞者的精确及小丑的滑稽混合在一起。如果你未获得,你同时也能得到啤酒酿造工具的熟练。

\subsubsection{醉拳窍 Drunken Technique}
第3级时,你从这个范型中学会了如何在疾风连打中快速的扭转与行动躯体。每当你使用疾风连击时,你将获得撤离动作的效果,而你的步行速度增加10尺直到当前回合结束。

\subsubsection{酣荡步 Tipsy Sway}当你6级时,你可以突然地以摇摆的方式作出移动。你将获得以下增益:

\subparagraph{鲤鱼打挺}当你倒地时,你可以花费5尺移动力起身而不是一半的移动力。

\subparagraph{借力打力}当一个生物对你的一次近战攻击失手时,你可以花费1点气作此反应,让这次攻击击中在你周围5尺范围内由你选择的一名可见生物,但不能是攻击者。

\subsubsection{酩酊运 Drunkard's Luck}
第6级起,你似乎总是能在最佳时刻获得机运的反馈。当你进行一次属性检定,攻击检定或豁免检定并在检定中拥有劣势时,你可以花费2点气抵消这个劣势。

\subsubsection{为酒狂 Intoxicated Frenzy}
第17级时,你能对一大群敌人进行超量次数的乱舞打击。 每当你使用你的疾风连击时,你可以用它进行最多三次的额外攻击(疾风连击最多可以进行五次攻击),前提是每一次疾风连击都能在本回合中取一个不同生物作为攻击目标。

\subsection{剑圣宗 Way of the Kensei}研习剑圣之道的武僧沉浸于器械训练,达到了“武器就是身体的延伸”这一境界。在研习剑术的基础上,这一流派也同时涉猎诸多不同的武器。

器械之于剑圣,就如同画笔之于画家、笔墨之于文人。剑圣将其视为用来展现武道之美丽优雅的工具。这一信条使得剑圣成为了盖世无双的武者,但与此同时也受缚于热情奉献和刻苦专研。

\begin{dndtable}[cX]
\textbf{等级} & \textbf{特性} \\
3rd & \emph{剑圣途 Path of the Kensei(两种武器)}\\
6th & \emph{\specialcell{人剑一 One with the Blade,\\剑圣途 Path of the Kensei(三种武器)}}\\
11th & \emph{\specialcell{利锋影 Sharpen the Blade,\\剑圣途 Path of the Kensei(四种武器)}}\\
17th & \emph{\specialcell{精绝剑 Unerring Accuracy,\\剑圣途 Path of the Kensei(五种武器)}}\\
\end{dndtable}

\subsubsection{剑圣途 Path of the Kensei}当第三级时,你特殊的武术训练使你精通于某些特定武器使用技巧。这一流派同时还包括了对书画艺术的研究。你获得以下好处:
\subparagraph{剑圣武器 Kensei Weapons.}选择两种武器作为你的剑圣武器:一种近战武器和一种远程武器。这两种武器可以在不带有重型和特殊属性的简易或军用武器中任选。长弓也算在内。如果你还未拥有所选武器的熟练项,你获得之。所选武器算作你的武僧武器。这一宗派的许多职业特性只有当你使用剑圣武器时才能生效。当你的武僧职业等级分别达到第6级、第11级和第17级时,你可以遵从上述标准,选择另一种武器——近战或远程武器均可——作为你的剑圣武器。
\subparagraph{灵巧格档 Agile Parry}如果你持握一件剑圣武器,并在你的回合中进行了一次徒手攻击作为攻击动作的一部分,若该武器是近战武器,则你可以用它来保护自己。直到你的下回合开始之前你的AC获得+2加值,只要武器还在你的手中或者你未陷入失能状态。
\subparagraph{持弓审固 Kensei's Shot}你可以在你的回合中使用一次附赠动作使你的剑圣武器远程攻击变得更为致命。当你这样做时,你用剑圣武器发动的远程攻击所命中的任何目标将会额外受到1d4点该武器的属性种类的伤害。你保留这一增益直到当前回合结束。
\subparagraph{书画之道 Way of the Brush}你在书法工具或画家工具中选择其一,获得熟练项。

\subsubsection{人剑一 One with the Blade}第6级时,你能将“气”灌注到剑圣武器之中,这为你带来以下好处:
\subparagraph{剑气 Magic Kensei Weapons.}攻击对非魔法攻击和伤害具有抗性或免疫的目标时,你的剑圣武器攻击视为魔法攻击。
\subparagraph{巧劲 Deft Strike}当你使用剑圣武器命中目标时,你可以消耗1点“气”使得该武器对该目标造成额外伤害,数值等于你的“武艺”骰。该特性你每回合只能使用一次。

\subsubsection{利锋影 Sharpen the Blade}
第11级时,你能使用“气”进一步地增强你的武器。你可以用一次附赠动作,消耗最多3点“气”灌注进你接触的一件剑圣武器,在你使用它攻击时获得攻击检定和伤害投骰的加值。加值等于你为此消耗的“气”点数。这一加值效果持续1分钟,除非你再次使用该特性。该特性无法作用于一件已经提供攻击检定与伤害骰加值的魔法武器。

\subsubsection{精绝剑 Unerring Accuracy}
第17级起,你对武器的运用更为精妙。如果你使用武僧武器进行的攻击未命中,你可以重新掷骰。该特性你每回合只能使用一次。

\subsection{日魂宗 Way of the Sun soul}日魂宗的武僧学习如何引导出自己的生命能量并将它们转换为炽热的光芒。他们相信在每一个生存的生物,灵魂中都有这样不熄的光芒存在,并且教导他们以一种特殊的冥想方式去解锁。

\begin{dndtable}[cX]
\textbf{等级} & \textbf{特性} \\
3rd & \emph{日辐箭 Radiant Sun Bolt}\\
6th & \emph{热能袭 Searing Arc Strike}\\
11th & \emph{阳光爆 Searing Sunburst}\\
17th & \emph{光耀盾 Sun Shield}\\
\end{dndtable}

\subsubsection{日辐箭 Radiant Sun Bolt}第3级起,选择了这个范型的武僧可以投掷出灼热的箭矢。

你可以在攻击动作中使用一种特殊的远程法术攻击(射程30尺)。你熟练于这次攻击,并且可以在攻击和伤害骰上加上敏捷调整值。它造成光耀伤害,伤害骰为1d4。伤害骰会随着你的等级增长而改变,参考武僧表中的武艺栏。

如果你在自己的回合内用攻击动作使用这种特殊攻击,你可以花费1点气,并且再用附赠动作用阳光箭进行两次额外的攻击。

当你得到额外攻击特性时,攻击中任意一次符合条件的攻击即可触发该特殊攻击。
\subsubsection{热能袭 Searing Arc Strike}第6级起,武僧可以转换自己的气变为灼热的能量。

如果你在你的回合内进行了攻击,你可以立即消耗2点气来用附赠动作施放一环\emph{燃烧之手 burning hands}。你可以花费额外的气来施展更高环的\emph{燃烧之手 burning hands},每一点额外的气都使法术等级上升1环。你最多可以在这个法术上花费的气(2+额外的点数)等同你一半的武僧等级(向下取整)。
\subsubsection{阳光爆 Searing Sunburst}从11级起,武僧可以创造出一个纯粹的光能量球以引起一场毁灭性的爆炸。

以一个动作,你可以魔法般地创造一个球体并且向150尺内的任意一点投掷,在那里瞬间扩张成一个光芒四射的区域。以目标点为中心半径20尺内的所有生物必须立即进行一次对抗气豁免的体质检定,豁免失败者将受到2d6的光耀伤害。若生物处于全掩护环境并且掩体不透光,那么他不需要进行豁免。

你可以花费额外的气以提高这个能量球的威力。每花费1点可以使伤害上升2d6,上限3点。
\subsubsection{光耀盾 Sun Shield}从17级起,武僧可以完全将自己包覆在魔法性的强光之中。

你发散出30尺的明亮照明和额外30尺的昏暗照明。你可以以一个附赠动作将这个光芒熄灭或者恢复。

如果一个被光芒照耀的生物对你进行近战攻击并击中了你,你可以用你的反应对这个生物造成5+你的感知调整值的光耀伤害。